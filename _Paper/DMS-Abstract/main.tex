\documentclass{article}
% \usepackage[utf8]{inputenc}
\usepackage[margin=1.0in]{geometry}
% \usepackage{soul} % strikeout \st{}
% \usepackage{graphicx}
% \usepackage{amsmath,amssymb,amstext}
% \usepackage{newtxtext,newtxmath}
\usepackage[dvipsnames]{xcolor}
\usepackage{url}
\usepackage[normalem]{ulem} % strikeout \sout{}

%%%%%%%%%%%%%%%%%%%%%%%%%%%%%%%%%%%%%%%%%%%%%%%%%%%%%%%%%%%%%%%%%%%%%%
%
%               Macros for TeX/LaTeX documents
%
%
%%%%%%%%%%%%%%%%%%%%%%%%%%%%%%%%%%%%%%%%%%%%%%%%%%%%%%%%%%%%%%%%%%%%%%
\usepackage{xspace} % smart spaces after custom \newcommand

%-- Comments and questions
\newcommand{\arz}[1]{{\color{ForestGreen}\textbf{[Andrew: }\textbf{#1}]}}
\newcommand{\cb}[1]{{\color{TealBlue}\textbf{[Carlos: }\textbf{#1}]}}
\newcommand{\hrm}[1]{{\color{Mulberry}\textbf{[Héctor: }\textbf{#1}]}}
\newcommand{\tjh}[1]{{\color{Red}\textbf{[Travis: }\textbf{#1}]}}
% \newcommand{\fvdb}[1]{{\color{BurntOrange}\textbf{[Frank: }\textbf{#1}]}}
% \newcommand{\cs}[1]{{\color{Lavender}\textbf{[ChadS: }\textbf{#1}]}}
\newcommand{\kw}[1]{{\color{Mulberry}\textbf{[Kuan: }\textbf{#1}]}}
\newcommand{\tjr}[1]{{\color{Brown}\textbf{[Troy: }\textbf{#1}]}}

\newcommand{\cbq}[0]{{\color{TealBlue}\textbf{[Question for Carlos: ]}}}
\newcommand{\todo}[1]{{\color{Mulberry}\textbf{[TO DO: }\textbf{#1}]}}
\newcommand{\qus}[1]{{\color{BrickRed}\textbf{[Q: }\textbf{#1}]}}


%-- Units
\newcommand{\cmsq}{\ \mathrm{cm}^2}
\newcommand{\cmcinv}{\ \mathrm{cm}^{-3}}
\newcommand{\gev}{\ \mathrm{GeV}}
\newcommand{\K}{\ \mathrm{K}}
\newcommand{\kms}{\ \mathrm{km\ s}^{-1}}
\newcommand{\Msun}{\ \mathrm{M}_{\odot}} % Msun
\newcommand{\sinv}{\ \mathrm{s}^{-1}}


%-- Variables
% must be used within math mode unless command includes $$
% \newcommand{\mp}{$m_p$}
\newcommand{\Csun}{C_{\odot}}
\newcommand{\Cx}{C_{\mathrm{DM}}}
\newcommand{\epsCNO}{\epsilon_{\mathrm{CNO}}}
\newcommand{\epsnuc}{\epsilon_{\mathrm{nuc}}}
\newcommand{\epspp}{\epsilon_{\mathrm{pp}}}
\newcommand{\epsx}{\epsilon_{\mathrm{DM}}}
\newcommand{\gb}{\Gamma_\mathrm{B}}
\newcommand{\gbzero}{\Gamma_\mathrm{B} = 0}
\newcommand{\gbone}{\Gamma_\mathrm{B} = 1}
\newcommand{\gbpow}[1]{\Gamma_\mathrm{B} = 10^{#1}}
\newcommand{\LvsT}{log($L$)-log($\Teff$)\xspace}
\newcommand{\lx}{l_{\mathrm{DM}}}
\newcommand{\mprot}{m_{\mathrm{p}}}
\newcommand{\mrange}{$0.9 \leq M_{\star}/\mathrm{M}_{\odot} \leq 5.0$\xspace}
\newcommand{\mrangelow}{$0.9 \leq M_{\star}/\mathrm{M}_{\odot} \lesssim 1.3$\xspace}
\newcommand{\mrangehigh}{$1.3 \lesssim M_{\star}/\mathrm{M}_{\odot} \leq 5.0$\xspace}
\newcommand{\Mstar}{M_{\star}}
\newcommand{\mx}{m_{\mathrm{DM}}}
\newcommand{\nprot}{n_{\mathrm{p}}}
\newcommand{\Nx}{N_{\mathrm{DM}}}
\newcommand{\nx}{n_{\mathrm{DM}}}
\newcommand{\rhox}{\rho_{\mathrm{DM}}}
\newcommand{\Rstar}{R_{\star}}
\newcommand{\rx}{r_{\mathrm{DM}}}
\newcommand{\sigxn}{\sigma_{\mathrm{n}}}
\newcommand{\sigxp}{\sigma_{\mathrm{p}}}
\newcommand{\Tc}{T_\mathrm{c}}
\newcommand{\Teff}{T_{\mathrm{eff}}}
\newcommand{\Tx}{T_{\mathrm{DM}}}

%-- Other
\newcommand{\mesa}{\texttt{MESA}\xspace}
\newcommand{\mist}{\texttt{MIST}\xspace}
\newcommand{\nodm}{\texttt{NoDM}\xspace}



%----------------------------------------
\title{Dark Matter in Stars}
\author{Troy J. Raen}
\date{May 2019}

\begin{document}

\maketitle


% ------------------- Talk Abstract -------------------
\begin{abstract}

  Our work explores the effects of energy transport by spin-dependent Asymmetric Dark Matter (ADM) in stars of mass \equin{0.8 \leq \Mstar/\Msun \leq 5.0} using the publicly available code Modules for Experiments in Stellar Astrophysics (\mesa). If dark matter (DM) can scatter with standard model particles, it can be captured by stars orbiting in DM halos and contribute to stellar energy transport via continued scattering. Despite ongoing terrestrial direct detection experiments, DM has not been observed and the available parameter space for classic WIMPs (Weakly-Interacting Massive Particles) is shrinking rapidly, prompting research into other models. ADM predicts a DM particle that is less massive than typical WIMPs and does not self-annihilate - a scenario that allows a star to build up a large quantity of DM with orbital paths crossing distinct regions of the star. Thus, the energy transported by DM can alter the stellar evolution. Our results show that, while the details vary as a function of stellar mass, the net effect on stars in rich DM environments is a significant, and potentially observable, decrease in stellar lifetimes. In this talk, I discuss the specific effects of DM energy transport on \mesa models and present results in the form of isochrones that may be used in the future to constrain DM properties by comparison with observed star clusters.

\end{abstract}

%---- Same as above without use of macros:
Our work explores the effects of energy transport by spin-dependent Asymmetric Dark Matter (ADM) in stars of mass $0.8 \leq \mathrm{M}_{\star}/\mathrm{M}_{\odot} \leq 5.0$ using the publicly available code Modules for Experiments in Stellar Astrophysics (\texttt{MESA}). If dark matter (DM) can scatter with standard model particles, it can be captured by stars orbiting in DM halos and contribute to stellar energy transport via continued scattering. Despite ongoing terrestrial direct detection experiments, DM has not been observed and the available parameter space for classic WIMPs (Weakly-Interacting Massive Particles) is shrinking rapidly, prompting research into other models. ADM predicts a DM particle that is less massive than typical WIMPs and does not self-annihilate - a scenario that allows a star to build up a large quantity of DM with orbital paths crossing distinct regions of the star. Thus, the energy transported by DM can alter the stellar evolution. Our results show that, while the details vary as a function of stellar mass, the net effect on stars in rich DM environments is a significant, and potentially observable, decrease in stellar lifetimes. In this talk, I discuss the specific effects of DM energy transport on \texttt{MESA} models and present results in the form of isochrones that may be used in the future to constrain DM properties by comparison with observed star clusters.


\end{document}
