\PassOptionsToPackage{pdfpagelabels=false}{hyperref}
% \documentclass[useAMS,fleqn,usenatbib]{mnras} % fleqn left-aligns equations
\documentclass[useAMS,usenatbib]{mnras}
\pdfoutput=1 %for arxiv
\setlength{\topmargin}{-0.3in}

\usepackage{graphicx}
\usepackage{amsmath,amssymb,amstext}
\usepackage[T1]{fontenc}
\usepackage{ae,aecompl}
\usepackage[utf8]{inputenc}
% \usepackage{newtxtext,newtxmath}
\usepackage[figure,figure*]{hypcap}
\usepackage[dvipsnames]{xcolor}
\usepackage{bm}

\usepackage{xparse}

%% MINE
\usepackage{xspace} % smart spaces after custom \newcommand



%----------------------------------------
\newcommand{\qus}[1]{{\color{BrickRed}\textbf{[Q: }\textbf{#1}]}}
\newcommand{\arz}[1]{{\color{ForestGreen}\textbf{[Andrew: }\textbf{#1}]}}
\newcommand{\jan}[0]{{\color{TealBlue}\textbf{[Jeff: ]}}}
\newcommand{\tjr}[1]{{\color{Brown}\textbf{[Troy: }\textbf{#1}]}}


%-- Units
\newcommand{\Msun}{\mathrm{M}_{\odot}} % Msun
\newcommand{\Mstar}{\mathrm{M}_{\star}} % Mstar
\newcommand{\Msunh}{h^{-1}\mathrm{M}_{\odot}} % Msun/h
\newcommand{\hmpc}{\mathrm{h}^{-1}\mathrm{Mpc}} % Mpc/h
\newcommand{\kms}{\mathrm{km/s}}
\newcommand{\gev}{\mathrm{GeV}}

%-- Words
\newcommand{\lcdm}{$\Lambda$CDM\xspace}
\newcommand{\photoz}{photo-$z$\xspace}
\newcommand{\photozs}{photo-$z$'s\xspace}
\newcommand{\Photozs}{Photo-$z$'s\xspace}
\newcommand{\specz}{spec-$z$\xspace}
\newcommand{\speczs}{spec-$z$'s\xspace}
\newcommand{\z}{$z$\xspace}
\newcommand{\Z}{$Z$\xspace}


%-- Code packages
\newcommand{\mesa}{\texttt{MESA}\xspace}
\newcommand{\fortran}{\texttt{Fortran}\xspace}
\newcommand{\python}{\texttt{Python}\xspace}
\newcommand{\rockstar}{\texttt{ROCKSTAR}\xspace}
\newcommand{\halotools}{\texttt{Halotools}\xspace}
\newcommand{\corrfunc}{\texttt{Corrfunc}\xspace}
\newcommand{\glue}{Glue\xspace}
\newcommand{\xgboost}{\texttt{XGBoost}\xspace}
\newcommand{\gpz}{\texttt{GPz}\xspace}




%-- Figures and Equations
\newcommand{\equ}[1]{\[#1\]}
\newcommand{\nequ}[2]{\begin{equation}#1 \label{#2}\end{equation}}
\newcommand{\equin}[1]{\(#1\)}
\newcommand{\figscale}[4]{
% commands: width scale, path, caption, label
\begin{figure}[h]
    \centering
    \includegraphics[width=#1\textwidth]{#2}
    \caption{#3}
    \label{#4}
\end{figure}
}


\usepackage{url}
\usepackage{hyphenat}

%% ENDMINE

%%%%%%%%%%%%%%%%%%% TITLE PAGE %%%%%%%%%%%%%%%%%%%

% Title of the paper, and the short title which is used in the headers.
% Keep the title short and informative.
\title[Asymmetric Dark Matter in Stars]{The Effects of Asymmetric Dark Matter on Stellar Evolution I: Spin\hyp{}Dependent Scattering}

% The list of authors, and the short list which is used in the headers.
% If you need two or more lines of authors, add an extra line using \newauthor
\author[T.J. Raen et al.]{%
Troy J. Raen,$^{1}$\thanks{E\hyp{}mail: troy.raen@pitt.edu},
Héctor Martínez\hyp{}Rodríguez$^{1}$,
Travis J. Hurst$^{2}$,
Andrew R. Zentner$^{1}$,\newauthor
Carles Badenes$^{1}$,
and Rachel Tao$^{3}$
\vspace*{12pt}
\\
% List of institutions
$^{1}$Department of Physics and Astronomy \& Pittsburgh Particle Physics, Astrophysics, and Cosmology Center (Pitt PACC),\\ University of Pittsburgh, Pittsburgh, PA 15260, USA\\
$^{2}$Department of Mathematics and Physics, Colorado State University \hyp{} Pueblo, Pueblo, C0, 81001 \\%Fixed my contact info. TJH
$^{3}$Department of Physics, Emory University, Atlanta, GA 30322
}
% These dates will be filled out by the publisher
\date{\today}

% Enter the current year, for the copyright statements etc.
\pubyear{2020}

% Don't change these lines
\begin{document}
\label{firstpage}
\pagerange{\pageref{firstpage}--\pageref{lastpage}}
\maketitle



% Abstract of the paper
\begin{abstract}
Most of the dark matter (DM) search over last few decades has focused on WIMPs but the viable parameter space is quickly shrinking. Asymmetric Dark Matter (ADM) is a WIMP-like DM candidate with slightly smaller masses and no present day annihilation, meaning that stars can capture and build up large quantities of it. The captured ADM can transport energy through a significant volume of the star. We investigate the effects of spin-dependent ADM energy transport on stellar structure and evolution in stars with \mrange in varying DM environments. We wrote a publicly available MESA module (footnote with repo link) that calculates the capture of DM and the subsequent energy transport within the star. We fix the DM mass to 5 GeV and the cross section to $10^{-37}$ cm${^2}$, and we study varying environments by scaling the DM capture rate. For stars with radiative cores ($\Mstar \lesssim 1.3 \Msun$), the presence of ADM flattens the temperature and burning profiles in the core, but the effect on MS lifetimes and observable properties is small. However, in higher mass stars, ADM energy transport shuts off the convection in the core, limiting the fuel available and therefore shortening MS lifetimes by as much as $~40\%$. This translates to changes in the luminosity and effective temperature of the MS turnoff in stellar population isochrones.
\end{abstract}

% Select between one and six entries from the list of approved keywords.
% Don't make up new ones.
\begin{keywords}
keyword1 -- keyword2 -- keyword3
\end{keywords}

%%%%%%%%%%%%%%%%%%%%%%%%%%%%%%%%%%%%%%%%%%%%%%%%%%







%%%%%%%%%%%%%%%%% BODY OF PAPER %%%%%%%%%%%%%%%%%%












%%%%%%%%%%%%%%%%%%%%%%%%%%%%%%%%%%%%%%%%%%%%%%%%%%



% Don't change these lines
\bsp	% typesetting comment

\label{lastpage}

\end{document}

% End of mnras_template.tex
