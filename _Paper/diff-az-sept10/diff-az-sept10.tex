\PassOptionsToPackage{pdfpagelabels=false}{hyperref}
%DIF LATEXDIFF DIFFERENCE FILE
%DIF DEL DM-in-Stars-AndrewDiffOG/main.tex   Thu Sep  3 18:10:50 2020
%DIF ADD DM-in-Stars-sept10/main.tex         Thu Sep 10 06:24:28 2020
% \documentclass[useAMS,fleqn,usenatbib]{mnras} % fleqn left-aligns equations
\documentclass[useAMS,usenatbib]{mnras}
\pdfoutput=1 %for arxiv
\setlength{\topmargin}{-0.3in}

\usepackage{graphicx}
\usepackage{amsmath,amssymb,amstext}
\usepackage[T1]{fontenc}
\usepackage{ae,aecompl}
\usepackage[utf8]{inputenc}
% \usepackage{newtxtext,newtxmath}
\usepackage[figure,figure*]{hypcap}
\usepackage[dvipsnames]{xcolor}
\usepackage{bm}
%DIF 16a16-19
% \usepackage{pdflscape} % rotate isochrones on page %DIF > 
\usepackage{subcaption} % put isos side by side %DIF > 
\usepackage{paralist} % inline lists %DIF > 
 %DIF > 
%DIF -------

\usepackage{xparse}

%% MINE
%%%%%%%%%%%%%%%%%%%%%%%%%%%%%%%%%%%%%%%%%%%%%%%%%%%%%%%%%%%%%%%%%%%%%%
%
%               Macros for TeX/LaTeX documents
%
%
%%%%%%%%%%%%%%%%%%%%%%%%%%%%%%%%%%%%%%%%%%%%%%%%%%%%%%%%%%%%%%%%%%%%%%
\usepackage{xspace} % smart spaces after custom \newcommand

%-- Comments and questions
\newcommand{\arz}[1]{{\color{ForestGreen}\textbf{[Andrew: }\textbf{#1}]}}
\newcommand{\cb}[1]{{\color{TealBlue}\textbf{[Carlos: }\textbf{#1}]}}
\newcommand{\hrm}[1]{{\color{Mulberry}\textbf{[Héctor: }\textbf{#1}]}}
\newcommand{\tjh}[1]{{\color{Red}\textbf{[Travis: }\textbf{#1}]}}
% \newcommand{\fvdb}[1]{{\color{BurntOrange}\textbf{[Frank: }\textbf{#1}]}}
% \newcommand{\cs}[1]{{\color{Lavender}\textbf{[ChadS: }\textbf{#1}]}}
\newcommand{\kw}[1]{{\color{Mulberry}\textbf{[Kuan: }\textbf{#1}]}}
\newcommand{\tjr}[1]{{\color{Brown}\textbf{[Troy: }\textbf{#1}]}}

\newcommand{\cbq}[0]{{\color{TealBlue}\textbf{[Question for Carlos: ]}}}
\newcommand{\todo}[1]{{\color{Mulberry}\textbf{[TO DO: }\textbf{#1}]}}
\newcommand{\qus}[1]{{\color{BrickRed}\textbf{[Q: }\textbf{#1}]}}


%-- Units
\newcommand{\cmsq}{\ \mathrm{cm}^2}
\newcommand{\cmcinv}{\ \mathrm{cm}^{-3}}
\newcommand{\gev}{\ \mathrm{GeV}}
\newcommand{\K}{\ \mathrm{K}}
\newcommand{\kms}{\ \mathrm{km\ s}^{-1}}
\newcommand{\Msun}{\ \mathrm{M}_{\odot}} % Msun
\newcommand{\sinv}{\ \mathrm{s}^{-1}}


%-- Variables
% must be used within math mode unless command includes $$
% \newcommand{\mp}{$m_p$}
\newcommand{\Csun}{C_{\odot}}
\newcommand{\Cx}{C_{\mathrm{DM}}}
\newcommand{\epsCNO}{\epsilon_{\mathrm{CNO}}}
\newcommand{\epsnuc}{\epsilon_{\mathrm{nuc}}}
\newcommand{\epspp}{\epsilon_{\mathrm{pp}}}
\newcommand{\epsx}{\epsilon_{\mathrm{DM}}}
\newcommand{\gb}{\Gamma_\mathrm{B}}
\newcommand{\gbzero}{\Gamma_\mathrm{B} = 0}
\newcommand{\gbone}{\Gamma_\mathrm{B} = 1}
\newcommand{\gbpow}[1]{\Gamma_\mathrm{B} = 10^{#1}}
\newcommand{\LvsT}{log($L$)-log($\Teff$)\xspace}
\newcommand{\lx}{l_{\mathrm{DM}}}
\newcommand{\mprot}{m_{\mathrm{p}}}
\newcommand{\mrange}{$0.9 \leq M_{\star}/\mathrm{M}_{\odot} \leq 5.0$\xspace}
\newcommand{\mrangelow}{$0.9 \leq M_{\star}/\mathrm{M}_{\odot} \lesssim 1.3$\xspace}
\newcommand{\mrangehigh}{$1.3 \lesssim M_{\star}/\mathrm{M}_{\odot} \leq 5.0$\xspace}
\newcommand{\Mstar}{M_{\star}}
\newcommand{\mx}{m_{\mathrm{DM}}}
\newcommand{\nprot}{n_{\mathrm{p}}}
\newcommand{\Nx}{N_{\mathrm{DM}}}
\newcommand{\nx}{n_{\mathrm{DM}}}
\newcommand{\rhox}{\rho_{\mathrm{DM}}}
\newcommand{\Rstar}{R_{\star}}
\newcommand{\rx}{r_{\mathrm{DM}}}
\newcommand{\sigxn}{\sigma_{\mathrm{n}}}
\newcommand{\sigxp}{\sigma_{\mathrm{p}}}
\newcommand{\Tc}{T_\mathrm{c}}
\newcommand{\Teff}{T_{\mathrm{eff}}}
\newcommand{\Tx}{T_{\mathrm{DM}}}

%-- Other
\newcommand{\mesa}{\texttt{MESA}\xspace}
\newcommand{\mist}{\texttt{MIST}\xspace}
\newcommand{\nodm}{\texttt{NoDM}\xspace}


\usepackage{url}

%% ENDMINE

%%%%%%%%%%%%%%%%%%% TITLE PAGE %%%%%%%%%%%%%%%%%%%

% Title of the paper, and the short title which is used in the headers.
% Keep the title short and informative.
\title[Asymmetric Dark Matter in Stars]{The Effects of Asymmetric Dark Matter on Stellar Evolution I: Spin-Dependent Scattering}

% The list of authors, and the short list which is used in the headers.
% If you need two or more lines of authors, add an extra line using \newauthor
\author[T.J. Raen et al.]{%
Troy J. Raen,$^{1}$\thanks{E-mail: troy.raen@pitt.edu},
Héctor Martínez-Rodríguez$^{1}$, 
Travis J. Hurst$^{2}$,
Andrew R. Zentner$^{1}$,\newauthor
Carles Badenes$^{1}$,
and Rachel Tao$^{3}$
\vspace*{12pt}
\\
% List of institutions
$^{1}$Department of Physics and Astronomy \& Pittsburgh Particle Physics, Astrophysics, and Cosmology Center (Pitt PACC),\\ University of Pittsburgh, Pittsburgh, PA 15260, USA\\
$^{2}$Department of Mathematics and Physics, Colorado State University - Pueblo, Pueblo, CO, 81001 \\%Fixed my contact info. TJH
$^{3}$Department of Physics, Emory University, Atlanta, GA 30322
}
% These dates will be filled out by the publisher
\date{\today}

% Enter the current year, for the copyright statements etc.
\pubyear{2020}




%%%%%%% !!!!!!!!!!!!!!!!!!!
% Using the following setting to turn off hyperlinks and supress the "\pdfendlink ended up on a different page" error described here:
% https://www.overleaf.com/learn/latex/Questions/What_does_%22%5Cpdfendlink_ended_up_in_different_nesting_level_than_%5Cpdfstartlink%22_mean%3F

\hypersetup{draft}

% Will need to turn this setting off after draft is complete.
%%%%%%% !!!!!!!!!!!!!!!!!!!





% Don't change these lines
%DIF PREAMBLE EXTENSION ADDED BY LATEXDIFF
%DIF UNDERLINE PREAMBLE %DIF PREAMBLE
\RequirePackage[normalem]{ulem} %DIF PREAMBLE
\RequirePackage{color}\definecolor{RED}{rgb}{1,0,0}\definecolor{BLUE}{rgb}{0,0,1} %DIF PREAMBLE
\providecommand{\DIFadd}[1]{{\protect\color{blue}\uwave{#1}}} %DIF PREAMBLE
\providecommand{\DIFdel}[1]{{\protect\color{red}\sout{#1}}}                      %DIF PREAMBLE
%DIF SAFE PREAMBLE %DIF PREAMBLE
\providecommand{\DIFaddbegin}{} %DIF PREAMBLE
\providecommand{\DIFaddend}{} %DIF PREAMBLE
\providecommand{\DIFdelbegin}{} %DIF PREAMBLE
\providecommand{\DIFdelend}{} %DIF PREAMBLE
\providecommand{\DIFmodbegin}{} %DIF PREAMBLE
\providecommand{\DIFmodend}{} %DIF PREAMBLE
%DIF FLOATSAFE PREAMBLE %DIF PREAMBLE
\providecommand{\DIFaddFL}[1]{\DIFadd{#1}} %DIF PREAMBLE
\providecommand{\DIFdelFL}[1]{\DIFdel{#1}} %DIF PREAMBLE
\providecommand{\DIFaddbeginFL}{} %DIF PREAMBLE
\providecommand{\DIFaddendFL}{} %DIF PREAMBLE
\providecommand{\DIFdelbeginFL}{} %DIF PREAMBLE
\providecommand{\DIFdelendFL}{} %DIF PREAMBLE
\newcommand{\DIFscaledelfig}{0.5}
%DIF HIGHLIGHTGRAPHICS PREAMBLE %DIF PREAMBLE
\RequirePackage{settobox} %DIF PREAMBLE
\RequirePackage{letltxmacro} %DIF PREAMBLE
\newsavebox{\DIFdelgraphicsbox} %DIF PREAMBLE
\newlength{\DIFdelgraphicswidth} %DIF PREAMBLE
\newlength{\DIFdelgraphicsheight} %DIF PREAMBLE
% store original definition of \includegraphics %DIF PREAMBLE
\LetLtxMacro{\DIFOincludegraphics}{\includegraphics} %DIF PREAMBLE
\newcommand{\DIFaddincludegraphics}[2][]{{\color{blue}\fbox{\DIFOincludegraphics[#1]{#2}}}} %DIF PREAMBLE
\newcommand{\DIFdelincludegraphics}[2][]{% %DIF PREAMBLE
\sbox{\DIFdelgraphicsbox}{\DIFOincludegraphics[#1]{#2}}% %DIF PREAMBLE
\settoboxwidth{\DIFdelgraphicswidth}{\DIFdelgraphicsbox} %DIF PREAMBLE
\settoboxtotalheight{\DIFdelgraphicsheight}{\DIFdelgraphicsbox} %DIF PREAMBLE
\scalebox{\DIFscaledelfig}{% %DIF PREAMBLE
\parbox[b]{\DIFdelgraphicswidth}{\usebox{\DIFdelgraphicsbox}\\[-\baselineskip] \rule{\DIFdelgraphicswidth}{0em}}\llap{\resizebox{\DIFdelgraphicswidth}{\DIFdelgraphicsheight}{% %DIF PREAMBLE
\setlength{\unitlength}{\DIFdelgraphicswidth}% %DIF PREAMBLE
\begin{picture}(1,1)% %DIF PREAMBLE
\thicklines\linethickness{2pt} %DIF PREAMBLE
{\color[rgb]{1,0,0}\put(0,0){\framebox(1,1){}}}% %DIF PREAMBLE
{\color[rgb]{1,0,0}\put(0,0){\line( 1,1){1}}}% %DIF PREAMBLE
{\color[rgb]{1,0,0}\put(0,1){\line(1,-1){1}}}% %DIF PREAMBLE
\end{picture}% %DIF PREAMBLE
}\hspace*{3pt}}} %DIF PREAMBLE
} %DIF PREAMBLE
\LetLtxMacro{\DIFOaddbegin}{\DIFaddbegin} %DIF PREAMBLE
\LetLtxMacro{\DIFOaddend}{\DIFaddend} %DIF PREAMBLE
\LetLtxMacro{\DIFOdelbegin}{\DIFdelbegin} %DIF PREAMBLE
\LetLtxMacro{\DIFOdelend}{\DIFdelend} %DIF PREAMBLE
\DeclareRobustCommand{\DIFaddbegin}{\DIFOaddbegin \let\includegraphics\DIFaddincludegraphics} %DIF PREAMBLE
\DeclareRobustCommand{\DIFaddend}{\DIFOaddend \let\includegraphics\DIFOincludegraphics} %DIF PREAMBLE
\DeclareRobustCommand{\DIFdelbegin}{\DIFOdelbegin \let\includegraphics\DIFdelincludegraphics} %DIF PREAMBLE
\DeclareRobustCommand{\DIFdelend}{\DIFOaddend \let\includegraphics\DIFOincludegraphics} %DIF PREAMBLE
\LetLtxMacro{\DIFOaddbeginFL}{\DIFaddbeginFL} %DIF PREAMBLE
\LetLtxMacro{\DIFOaddendFL}{\DIFaddendFL} %DIF PREAMBLE
\LetLtxMacro{\DIFOdelbeginFL}{\DIFdelbeginFL} %DIF PREAMBLE
\LetLtxMacro{\DIFOdelendFL}{\DIFdelendFL} %DIF PREAMBLE
\DeclareRobustCommand{\DIFaddbeginFL}{\DIFOaddbeginFL \let\includegraphics\DIFaddincludegraphics} %DIF PREAMBLE
\DeclareRobustCommand{\DIFaddendFL}{\DIFOaddendFL \let\includegraphics\DIFOincludegraphics} %DIF PREAMBLE
\DeclareRobustCommand{\DIFdelbeginFL}{\DIFOdelbeginFL \let\includegraphics\DIFdelincludegraphics} %DIF PREAMBLE
\DeclareRobustCommand{\DIFdelendFL}{\DIFOaddendFL \let\includegraphics\DIFOincludegraphics} %DIF PREAMBLE
%DIF LISTINGS PREAMBLE %DIF PREAMBLE
\RequirePackage{listings} %DIF PREAMBLE
\RequirePackage{color} %DIF PREAMBLE
\lstdefinelanguage{DIFcode}{ %DIF PREAMBLE
%DIF DIFCODE_UNDERLINE %DIF PREAMBLE
  moredelim=[il][\color{red}\sout]{\%DIF\ <\ }, %DIF PREAMBLE
  moredelim=[il][\color{blue}\uwave]{\%DIF\ >\ } %DIF PREAMBLE
} %DIF PREAMBLE
\lstdefinestyle{DIFverbatimstyle}{ %DIF PREAMBLE
	language=DIFcode, %DIF PREAMBLE
	basicstyle=\ttfamily, %DIF PREAMBLE
	columns=fullflexible, %DIF PREAMBLE
	keepspaces=true %DIF PREAMBLE
} %DIF PREAMBLE
\lstnewenvironment{DIFverbatim}{\lstset{style=DIFverbatimstyle}}{} %DIF PREAMBLE
\lstnewenvironment{DIFverbatim*}{\lstset{style=DIFverbatimstyle,showspaces=true}}{} %DIF PREAMBLE
%DIF END PREAMBLE EXTENSION ADDED BY LATEXDIFF

\begin{document}
\label{firstpage}
\pagerange{\pageref{firstpage}--\pageref{lastpage}}
\maketitle



% Abstract of the paper
\begin{abstract}
Most of the dark matter (DM) search over last few decades has focused on WIMPs but the viable parameter space is quickly shrinking. Asymmetric Dark Matter (ADM) is a WIMP-like DM candidate with slightly smaller masses and no present day annihilation, meaning that stars can capture and build up large quantities of it. The captured ADM can transport energy through a significant volume of the star. We investigate the effects of spin-dependent ADM energy transport on stellar structure and evolution in stars with \mrange in varying DM environments. We wrote a MESA module\DIFdelbegin \footnote{\DIFdel{url}%DIFDELCMD < {%%%
\DIFdel{https://github.com/troyraen/DM-in-Stars}%DIFDELCMD < }%%%
} %DIFAUXCMD
\addtocounter{footnote}{-1}%DIFAUXCMD
\DIFdelend \DIFaddbegin \footnotemark\xspace \DIFaddend that calculates the capture of DM and the subsequent energy transport within the star. We fix the DM mass to 5 GeV and the cross section to $10^{-37}$ cm${^2}$, and we study varying environments by scaling the DM capture rate. For stars with radiative cores ($\Mstar \lesssim 1.3 \Msun$), the presence of ADM flattens the temperature and burning profiles in the core and increases MS lifetimes \DIFaddbegin \DIFadd{($X_c > 10^{-3}$) }\DIFaddend by up to $\sim20\%$. In higher mass stars \DIFaddbegin \DIFadd{($1.3 \Msun \lesssim \Mstar \lesssim 4.0 \Msun$)}\DIFaddend , ADM energy transport shuts off the convection in the core, limiting the fuel available and therefore shortening MS lifetimes by as much as $\sim 40\%$. This may translate to changes in the luminosity and effective temperature of the MS turnoff in stellar population isochrones\DIFdelbegin \DIFdel{. The }\DIFdelend \DIFaddbegin \DIFadd{, with isochrones of $\sim 0.1$ Gyr having slightly hotter MS turnoffs, and those with $\sim 1$ Gyr having slightly cooler turnoffs. The tip of the red giant branch may occur at lower luminosities. The }\DIFaddend effects are largest in DM environments with high densities and/or low velocity dispersions, making dwarf and early forming galaxies most likely to display the effects.
\end{abstract}

\DIFaddbegin \footnotetext{\url{https://github.com/troyraen/DM-in-Stars}}

\DIFaddend % Select between one and six entries from the list of approved keywords.
% Don't make up new ones.
\begin{keywords}
keyword1 -- keyword2 -- keyword3
\end{keywords}

%%%%%%%%%%%%%%%%%%%%%%%%%%%%%%%%%%%%%%%%%%%%%%%%%%







%%%%%%%%%%%%%%%%% BODY OF PAPER %%%%%%%%%%%%%%%%%%


\section{Introduction}
\label{sec:intro}

  A preponderance of the evidence suggests that approximately $84\%$ of the matter budget of the 
  universe consists of a form of non-baryonic dark matter that has yet to be identified 
  \citep[e.g.,][]{jungman_etal96,Bertone+05,CosmicVisions17,Profumo+19}. 
  In the standard picture of cosmological structure formation, 
  galaxies form within the potential wells of large, 
  nearly virialized halos of dark matter \citep{white_rees78,blumenthal_etal84}. 
  If the dark matter interacts with standard model particles, 
  it can be captured by stars moving through dark matter halos 
  \citep{press_spergel85,krauss_etal85,gaisser_etal86,griest_seckel87}. 
  Once captured, continued scattering within the stellar interior contributes 
  to energy transport within the star, potentially altering its evolution \DIFdelbegin \DIFdel{\mbox{%DIFAUXCMD
\citep{Spergel1985EffectInterior,Taoso+10,Zentner2011AsymmetricDwarfs,Iocco+12,Lopes_Silk12,Casanellas_Lopes13,Casanellas+15,vincent_etal15,Murase_Shoemaker16,Lopes_silk19}}\hspace{0pt}%DIFAUXCMD
}\DIFdelend \DIFaddbegin \DIFadd{\mbox{%DIFAUXCMD
\citep{Spergel1985EffectInterior,Taoso+10,Zentner2011AsymmetricDwarfs,Iocco+12,Lopes_Silk12,Casanellas_Lopes13,Casanellas+15,vincent_etal15,Murase_Shoemaker16,Lopes_silk19,Vincent2020}}\hspace{0pt}%DIFAUXCMD
}\DIFaddend . 
  The significance of this energy transport depends on the following 
  properties (in addition to the properties of the star): 
  (1) the DM mass, $\mx$; 
  (2) the DM-nucleon scattering cross section, $\sigxn$; 
  and (3) the total number of DM particles captured by a star, $\Nx$, 
  which itself depends on $\mx$ and $\sigxn$ as well as the local DM environment from 
which the particles are captured (see \S~\ref{sec:props}). 
We study the effects of energy transport by asymmetric dark matter 
(ADM, see below)
in stars of mass \mrange living within a variety of dark matter 
environments using the publicly-available code 
Modules for Experiments in Stellar Astrophysics 
\citep[\mesa,][]{Paxton2011ModulesMESA, Paxton2013, Paxton2015, Paxton2018, Paxton2019}


  Evidence supporting the claim that $\sim$84\% of the matter in the universe is in 
  an unknown form of dark matter is abundant and varied, ranging from the 
  anisotropy of the microwave background radiation to formation and structures of galaxies 
  \citep[e.g.,][]{jungman_etal96,Bertone+05,PLanck18}. 
  For several decades, the leading candidate has been the so called 
  Weakly-interacting massive particle (WIMP). 
  The classic WIMP is a heavy ($\mx \sim 10^2-10^3 \gev$) 
  thermal relic whose contemporary abundance is set 
  by its annihilation rate in the early universe 
  \cite[e.g.,][]{kolb_turner90}. 
  Therefore, WIMPs are thought to have a fairly well established ``standard'' annihilation 
  cross section \citep[e.g.,][]{steigman_etal12}, which is comparable to typical weak-scale 
  cross sections, $\langle \sigma v \rangle \sim 10^{-26} \, \mathrm{cm}^3/\mathrm{s}$. 
  This annihilation of WIMPs, which is so critical to guaranteeing that the 
  correct abundance of dark matter in the contemporary Universe, in turn, 
  limits the number of particles that can accumulate within a star. 
  The rate of capture of new dark matter particles 
  comes to equilibrium with dark matter particle annihilation in the 
  stellar interior \citep{krauss_etal85}. 
  Despite numerous ongoing terrestrial direct detection experiments 
  \citep[see][for a review]{Schumann19} 
  and efforts to detect dark matter indirectly through 
  its annihilation products \citep[reviewed in][]{Slatyer17}, 
  dark matter has not been observed non-gravitationally. The 
  available parameter space for relatively light ($\mx \lesssim 10^2 \mathrm{GeV}$) DM
  is rapidly shrinking, which has triggered a surge in research into 
  alternatives to the long-favored WIMP.


  Asymmetric dark matter (ADM) is an alternative to the classic WIMP in which 
  the relic abundance of the dark matter particle is set by a primordial asymmetry 
  rather than via annihilation \citep[for a review, see][and references therein]{adm_review}. 
  If the baryon and dark matter asymmetries are 
  related, then such models have the appealing property that they explain 
  the fact that the contemporary dark matter and baryon abundances are 
  of the same order of magnitude, which is otherwise surprising because 
  these relic abundances are determined by unrelated physics in the WIMP 
  scenario. Indeed, this was one of the early motivations for ADM-like 
  models \citep[e.g.,][]{Nussinov85,barr_etal90,chivukula_walker90,kaplan92}. 
  The variety of specific incarnations of ADM is broad, 
  but ADM models often predict particle masses smaller than 
  the classic WIMP ($\mx \sim 1-10 \gev$) and little or no 
  contemporary dark matter annihilation for lack of relic 
  dark matter anti-particles. 

  
  These predictions motivate studies to 
  constrain ADM indirectly through stellar astrophysics. The lack of 
  annihilation means that ADM may build up to very large
  quantities within stars because the capture of ADM is never countered 
  by annihilation. Meanwhile, the relatively low masses 
  compared to the classic WIMP mean captured ADM particles orbit within 
  a significant volume of the star, out to $\rx \sim 0.1 R_{\odot}$
  for a Sun-like star, which means that they experience large differences 
  in ambient temperature 
  throughout their orbits and can thus transport energy outward from the 
  stellar core extremely efficiently \citep{Spergel1985EffectInterior}. 
  These features of ADM have already motivated research into the possibility that 
  ADM may alter stellar evolution 
  \DIFdelbegin \DIFdel{\mbox{%DIFAUXCMD
\citep[e.g.][]{Taoso+10,Zentner2011AsymmetricDwarfs,Iocco+12,Lopes_Silk12,Casanellas_Lopes13,Casanellas+15,vincent_etal15,Murase_Shoemaker16,Lopes_silk19}}\hspace{0pt}%DIFAUXCMD
}\DIFdelend \DIFaddbegin \DIFadd{\mbox{%DIFAUXCMD
\citep[e.g.][]{Taoso+10,Zentner2011AsymmetricDwarfs,Iocco+12,Lopes_Silk12,Casanellas_Lopes13,Casanellas+15,vincent_etal15,Murase_Shoemaker16,Lopes_silk19,Vincent2020}}\hspace{0pt}%DIFAUXCMD
}\DIFaddend .
  In this paper we undertake a study of the properties and evolution of 
  stellar populations within halos of ADM. In this first paper on the topic, 
  we further narrow our study to spin-dependent ADM-nucleus scattering. 
  Spin-independent ADM-nucleus scattering leads to behaviors that are 
  qualitatively distinct from spin-dependent scattering; therefore we will 
  present results for the former case in a forthcoming manuscript. 

  We generally find that ADM captured by stars can cool 
  stellar cores to a degree that can have potentially 
  observable effects on stellar populations. In general, 
  the extra cooling due to ADM reduces the main sequence (MS) lifetimes 
  of stars with masses between $\sim 1.2 M_{\odot}$ and $\sim 5 M_{\odot}$, 
  and alters their subsequent evolution. Stars with masses between 
  $0.9 M_{\odot}$ and $1.2 M_{\odot}$ have their lifetimes altered in 
  complicated manner that is not a monotonic function of the model parameters. 
  We summarize our results on stellar lifetimes in Figure~\ref{fig:mstau} 
  and the net effect on stellar populations in the form of stellar isochrones in
  \S~\ref{sec:results}. The effects of stellar cooling are particularly large in 
  environments in which 
  the ambient dark matter density is high and 
  velocity dispersion is low, such that the capture of 
  dark matter is extremely efficient. 
  Thus, these effects will be largest in dwarf satellite galaxies 
  and high-redshift galaxies. 

  In the following section, we summarize the dependence of the capture rate of dark matter 
  within stars on both dark matter and stellar properties. In \S~\ref{sec:methods}, we describe
  our simulations of stellar evolution including cooling due to ADM. We present our results 
  in \S~\ref{sec:results}. We discuss our results and draw conclusions in \S~\ref{sec:discus}.


%----------------------------------------------
\section{Dark Matter Properties and Capture in Stars}
\label{sec:props}

  Probing the parameter space of ADM with simulations of stellar evolution is computationally expensive. 
  Consequently, we show results for a representative set of 
  ADM parameters that we initially chose in order to: 
  (1) make the effects of ADM on stellar evolution significant; 
  and (2) remain consistent with contemporary constraints on 
  dark matter properties (but see the discussion in the next 
  paragraph). For our models we choose $\mx = 5 \gev$ and a 
  spin-dependent dark matter--nucleon scattering cross section of 
  $\sigxp = 10^{-37} \cmsq$.
  Hereafter we will discuss ADM-proton scattering since protons 
  are the only nuclei in MS stars with both 
  a significant abundance and a net spin.
  We assume that ADM self-interactions 
  are negligible throughout; however, 
  it is likely that self-interactions, if they existed, 
  would lead to enhanced cooling 
  \citep[e.g.,][]{Zentner2009High-energySun} and exploring such models 
  would constitute a potentially interesting follow-up to this work.

  While we were performing the numerical simulations described in this 
  work, the PICO collaboration was able to reduce its thresholds, 
  unambiguously excluding dark matter with the specific parameter 
  values listed above \citep{PICO}. Unfortunately, the simulations 
  that we have carried out are computationally intensive and it is 
  impractical to repeat each of the $\sim 600$ stellar evolution 
  simulations. Consequently, we choose to present these results 
  as a qualitative indication of the effects that ADM can have 
  upon stellar evolution. As we will see below, there is 
  significant uncertainty involved in associating a particular 
  stellar effect with a particular dark matter-nucleon scattering 
  cross section due to a variety of model uncertainties. 
  Consequently, while the effects of dark matter within stars 
  will likely be milder than those that we describe here, it is 
  possible that the same, qualitative effects could be realized 
  in nature.

The amount of energy transported by dark matter 
(see \S~\ref{sub:energytransport}) is proportional 
to the amount of ADM within the star. 
In ADM models, in which annihilation of dark matter is negligible, 
the number of dark matter particles within the star at any given time, 
$t$, is determined by ${\rm d}\Nx/{\rm d}t = \Cx$, 
where $\Cx$ is the instantaneous ADM capture rate. 
We use the capture rate from \citet{Zentner2011AsymmetricDwarfs}, 
which is a simplified form valid for dark matter particle masses 
$\mx \lesssim 20 \gev$ \DIFdelbegin \DIFdel{\mbox{%DIFAUXCMD
\citep[see][ for more complete capture rates]{Gould1992CosmologicalAnnihilations,Zentner2009High-energySun}}\hspace{0pt}%DIFAUXCMD
}\DIFdelend \DIFaddbegin \DIFadd{\mbox{%DIFAUXCMD
\citep[see][for more complete capture rates]{Gould1992CosmologicalAnnihilations,Zentner2009High-energySun}}\hspace{0pt}%DIFAUXCMD
}\DIFaddend :
%
  \begin{align}
  \begin{split}
    \label{eq:capturerate}
    \Cx =\ & \Csun
    \Big(\frac{\rhox}{0.4 \gev \cmcinv}\Big)
    \Big(\frac{270 \kms}{\bar{v}}\Big) \\
    & \times \Big(\frac{\sigxp}{10^{-43} \cmsq}\Big) \Big(\frac{5 \gev}{\mx}\Big) \\
    & \times \Big(\frac{v_{\mathrm{esc}}}{618 \kms}\Big)^2
    \Big(\frac{\Mstar}{\Msun}\Big)
  \end{split}
  \end{align}
%
  where $\Csun = 5 \times 10^{21} \sinv$, 
  $\rhox$ is the DM density in the stellar environment,
  $\bar{v}$ is the velocity dispersion of dark matter particles 
  in the stellar neighborhood, and $v_{\mathrm{esc}}$ is the escape speed from the 
  surface of the star.


  The first line of Eq.~(\ref{eq:capturerate}) gives 
  the dependence of the capture rate on the stellar environment. 
  Both $\rhox$ and $\bar{v}$ are properties of the local stellar environment 
  and are degenerate with one another in Eq.~(\ref{eq:capturerate}). 
  A higher ambient density of dark matter leads to a higher rate of dark matter capture, 
  while a lower relative velocity between the star and the infalling dark matter leads 
  to a higher probability for capture. 
  Because of this degeneracy, coupled with the fact that these parameters carry 
  considerable uncertainty themselves, 
  it is convenient to parameterize a star's 
  local dark matter environment by an overall 
  factor \citep{Zentner2011AsymmetricDwarfs,Hurst2015},
%
  \begin{equation}
  \gb = \bigg(\frac{\rhox}{0.4 \gev \cmcinv}\bigg) \Bigg(\frac{270 \kms}{\bar{v}}\Bigg).
  \label{eq:gammab}
  \end{equation}
%
\DIFdelbegin %DIFDELCMD < \arz{Should we just call it ``$\Gamma$''? I don't really see a reason for the 
%DIFDELCMD < subscript now that I'm looking at it.}
%DIFDELCMD < %%%
\DIFdelend %DIF >  \arz{Should we just call it ``$\Gamma$''? I don't really see a reason for the 
%DIF >  subscript now that I'm looking at it.}
Normalized in this way, $\gb$ specifies the capture rate, 
$\Cx$, relative to the rate that would be realized in the 
solar neighborhood for the same star. 
From this point on, 
we will characterize a star's dark matter environment using 
$\gb$. In general, we will be most interested in values of $\gb > 1$. 
A value of $\gbzero$ describes a stellar environment with no dark matter 
(hereafter referred to as `standard models' and labeled `\nodm'), 
and $\gbone$ describes the solar neighborhood. 
A value of $\gbpow{2}$ may specify an environment in 
which the dark matter density is 100 times that in the 
solar neighborhood at the same velocity dispersion, 
an environment in which the velocity dispersion is 1/100 that 
of the solar neighborhood at the same density, 
or any of an infinite number of other possible combinations. 

 
It is interesting to consider the range of $\gb$ values 
that would be considered reasonable. If the distribution of 
dark matter within galaxies, such as the Milky Way, follows a 
profile that diverges as the 
\citet[][NFW]{nfwprofile} density profile, then one might expect 
to find dark matter environments near the centers of galaxies with
densities significantly higher than the local value and velocity dispersions significantly
lower than the local value, giving $\gb \gg 1$. Such scenarios 
were explored in \citet{Bertone_Fairbairn08}, \citet{Fairbairn+08}, 
and \citet{Scott+09}. While such large values of $\gb$ may well 
lead to large effects on stellar structure, 
stellar populations near the Galactic Center
are difficult to observe and any assumption 
about the dark matter density profile in the 
inner regions of any galaxy must be considered speculative. 
Interestingly, Local Group dwarf galaxies are extremely dark matter-dominated 
and have well-constrained dark matter profiles and velocity dispersions. 
In some cases, the Local Group dwarfs have densities $\sim 3$ orders 
of magnitude higher than the dark matter density in the Solar neighborhood 
and velocity dispersions that are at least $\sim 2$ orders of magnitude 
smaller than the local value. This suggests that values of $\gb \sim 10^{5-6}$ 
could be realized within Local Group dwarf galaxies and has the further merit that 
$\gb$ within Local Group dwarfs can be measured more precisely in the future. 
A third possibility for large values of the environmental factor 
are early-forming, very high-redshift galaxies. These galaxies begin 
forming in small, dense halos where the environmental boost factor 
can reach $\gb \gtrsim 10^6$ at redshifts $z \gtrsim 10$ \citep{KBD}. 
Of course, these stars will not be directly observable, but it 
is interesting to speculate that such stars could be detected as 
remnants of early mergers with the proto-Milky Way and/or that 
changes to the structure and evolution of these stars could be detected 
indirectly in the chemical evolution of the larger, lower-redshift 
galaxies in which they will be found today. 


Finally, while we have focused on the environmental parameter, $\gb$, 
as a proxy for the dark matter environment in which a star is embedded, 
we note that values of $\gb \ne 1$ can also be mimicked through dark matter 
physics. In particular, dark matter self-interactions can greatly 
enhance the capture rates of dark matter within 
stars \citep{Zentner2009High-energySun}. This effect of dark matter 
self-capture itself grows with increasing ambient density and 
decreasing ambient velocity dispersion, so the two effects 
reinforce one another. For example, a value of $\gb \sim 10^4$ may 
be realized by increasing the ambient dark matter density by a 
factor of $\sim 10^3$, while simultaneously introducing a 
dark matter self-interaction that boosts the number of 
captured dark matter particles by a factor of $\sim 10$. 
We relegate the separation of these effects to future work 
and encapsulate all of this uncertainty into the single 
parameter $\gb$. 


%----------------------------------------------
\section{Methods}
\label{sec:methods}

We study the impact of dark matter on the evolution of \mrange stars \DIFdelbegin \DIFdel{through the thermal pulse (or equivalent) phase }\DIFdelend \DIFaddbegin \DIFadd{(with a mass step of $0.05 \Msun$) through core helium depletion ($Y_c = 10^{-3}$) or a maximum age of 10 Gyr, whichever comes first, }\DIFaddend using the publicly-available code 
Modules for Experiments in Stellar Astrophysics \citep[\mesa,][]{Paxton2011ModulesMESA, Paxton2013, Paxton2015, Paxton2018, Paxton2019}, 
release $12115$. 
We used the \texttt{MESA SDK} version $20190830$\footnote{\url{https://zenodo.org/record/3560834}} to compile \mesa.
% \mesa models stellar evolution by simultaneously solving coupled differential equations that describe stellar structure and composition.  
We base our stellar parameter inlist on the \mesa test suite \texttt{1M\_pre\_ms\_to\_wd} inlists and use a metallicity of $Z = 0.0142$. \DIFaddbegin \DIFadd{12 of the models we ran did not complete (e.g., due to requiring unreasonably small timesteps), and we have excluded them from our final data set. Of these, 2 were $\gbpow{4}$ models, and none of them were either }\nodm \DIFadd{or $\gbpow{6}$ models (these are the three $\gb$ values we highlight below). }\DIFaddend We wrote a module that calculates dark matter capture and energy transport (see \S~\ref{sub:energytransport}) and connects to \mesa simulations via the provided \texttt{extra\_energy\_implicit} hook. Our code is available on GitHub\footnote{\url{https://github.com/troyraen/DM-in-Stars}} and through the \mesa Marketplace\footnote{\url{http://cococubed.asu.edu/mesa_market/add-ons.html}}.

To generate isochrones from \mesa's stellar models we must interpolate the data because time steps are adaptive and therefore data is not generated at a consistent set of ages.
We attempted to use the \DIFdelbegin \texttt{\DIFdel{MIST}} %DIFAUXCMD
\DIFdelend \DIFaddbegin \mist \DIFaddend code \citep[\mesa Isochrones and Stellar Tracks,][]{Dotter2016MesaIsochrones, Choi2016MESAModels}, which uses a multi-step process to interpolate both the stellar tracks and the mass grid and therefore produces isochrones that are well-sampled, even through dynamic but short lived phases such as the sub-giant branch. However, this method failed to produce isochrones in much of our parameter space of interest\DIFaddbegin \DIFadd{, including producing no isochrones older than 1 Gyr}\DIFaddend . This is likely due to non-monotonicity in the relation between initial stellar mass and age of a given evolutionary phase, which is present in our data (see \S~\ref{sub:mstau}) but violates assumptions of the code. Instead we simply \DIFdelbegin \DIFdel{interpolate }\DIFdelend \DIFaddbegin \DIFadd{perform a linear interpolation of }\DIFaddend our stellar tracks to a uniform set of ages and choose isochrones that are as well-sampled as possible in regions of interest. \DIFaddbegin \mesa\DIFadd{'s adaptive time steps resolve dynamic phases of evolution quite well, and this interpolation is not problematic. We do not interpolate the mass grid. 
}\DIFaddend 


%---------
\subsection{Energy Transport by Dark Matter}
\label{sub:energytransport}

  The energy transported by captured ADM can, in principle, be computed by solving the Boltzmann equation; however, this strategy is too computationally intensive to combine with a full-scale simulation of the evolution of stellar structure. To reduce the computational costs of our simulations, we estimate ADM energy transport using the approximations of \citet{Spergel1985EffectInterior}. In particular, we assume a Maxwellian phase-space distribution for the ADM and calculate an orbit-averaged temperature, $\Tx$, by requiring that the distribution satisfy the first moment of the Boltzmann equation. This amounts to a requirement on energy conservation: ADM should neither inject nor remove a net energy from the star. The rate of energy transfer (per unit mass) from dark matter to protons is then
  % \tjr{use \ is full space, '\,' '\.' . can also use begin{eqnarray} }
  %
  \begin{align}
  \begin{split}
  \epsx(r) =\ & 8\ \sqrt[]{\frac{2}{\pi}} \frac{\nx(r) \nprot(r)}{\rho(r)} \frac{\mx \mprot}{(\mx+\mprot)^2} \sigxp\\
  & \times \left(\frac{\mprot k \Tx + \mx k T(r)}{\mx \mprot}\right)^{1/2} k [\Tx - T(r)],
  \label{eq:xheat}
  \end{split}
  \end{align}
  %
  where $n(r)$ is a number density, $\rho(r)$ is the mass density, $k$ is Boltzmann's constant, and the subscript p refers to protons. \citep[See][for a detailed derivation]{Spergel1985EffectInterior}.

  Generally, $\nprot$, $\nx$, and $T$ all peak at the center (exceptions noted below), so the energy transport is most efficient here. The number density 
  of dark matter particles, $\nx$ increases in proportion with $\Nx$, so we can expect the effects to increase with both $\gb$ and stellar age through the MS, while hydrogen is abundant. As a star leaves the MS, $\nprot$ drops in the core and spin-dependent ADM energy transport is greatly diminished because there are relatively few protons left with which dark matter may scatter\footnote{This is one of the primary reasons that spin-dependent and spin-independent scattering give qualitatively distinct results. As the star burns H on the MS, the number of protons is reduced, reducing the importance of spin-dependent scattering processes. In the case of spin-independent scattering, the effect gets more important as helium 
  is produced from H burning during the MS.}. 
%   A standard MS temperature profile decreases monotonically with distance from the star's center, but it can become temporarily inverted when ADM moves large amounts of energy away from the center (requires $\gb \gg 1$)

  The sign of $\epsx(r)$ is given by the final term in (\ref{eq:xheat}), $\Tx - T(r)$, which is used to define an ADM characteristic radius, $\rx$, implicitly as
  %
  \begin{align}
    T(r=\rx) = \Tx.
  \end{align}
  %
  Then dark matter takes energy from $r < \rx$ and deposits it at $r > \rx$ for a standard MS temperature profile (monotonically decreasing from the center outward). With our chosen ADM parameters we see typical values:
  %
  \begin{align}
    \rx & \sim \mathcal{O}(0.1 \Rstar) \\
    \lx = (\sigxp \nprot)^{-1} & \sim\mathcal{O}(1 \Rstar)
  \end{align}
  %
  where $\lx$ is the ADM mean free path (implying that it completes several orbits between scattering events). These values allow dark matter to travel much larger distances than photons or ions within the star (which have $l \lesssim 10^{-10} \Rstar$) and to traverse qualitatively distinct 
  regions of the star. This large mean free path is what enables dark matter 
  to serve as such an effective coolant despite being far less numerous than either photons or ions \citep{Spergel1985EffectInterior}.

  % \arz{The preceding statement is probably not true, right? Near $r=0$, the temperature gradient is very small, so energy transport probably is not maximally efficient at the center of the star.} \tjr{What I mean is that the magnitude of the extra heat is largest at r=0, see Fig \ref{fig:m1p0_a}, top right plot. This is generally, but not always, true. Is it the word "efficient" that's a problem?}


%------------------------------------------
\section{Results}
\label{sec:results}

In standard stellar evolution, with no influence from dark matter, stars with \mrange naturally split into two groups with qualitatively different structures, based on the dominant channel through which they burn hydrogen. Spin-dependent asymmetric dark mater (ADM) affects core hydrogen burning, mainly by flattening the temperature gradient. In this section, we will review standard stellar astrophysics \citep{Kippenhahn2012} and then describe the effects of asymmetric dark matter seen in our MESA simulations.

The dominant burning channel is determined by the core temperature, 
with the transition happening at $\Tc \sim 2 \times 10^7 \K$, 
which corresponds to a stellar mass of $\Mstar \sim1.3 \Msun$.
Stars with $\Mstar \lesssim 1.3 \, \Msun$, which we call low-mass stars, 
burn hydrogen primarily through the proton-proton (pp) chain 
for which the burning rate scales with temperature very roughly as $\epspp \propto T^4$. 
For stars in the mass range $0.4 \lesssim \Mstar/\Msun \lesssim 1.3$, 
the transport of energy away from the core burning region is dominated 
by photon diffusion. Energy transport in the cores of such stars is said 
to be radiative.

Stars with $\Mstar \gtrsim 1.3 \Msun$, 
which we call high-mass stars, 
are dominated by the carbon-nitrogen-oxygen (CNO) cycle, 
for which the burning rate scales much more strongly with temperature, 
$\epsCNO \propto T^{16-20}$. In CNO-dominated stars, 
radiative energy transport is insufficient 
to carry away the energy produced by hydrogen burning; consequently, they have convective cores.


In \S~\ref{sub:lowmass} and \S~\ref{sub:highmass}, we will consider results for low-mass stars and high-mass stars separately and we will demonstrate that ADM has distinct effects on the evolution of the two groups. \S~\ref{sub:mstau} details changes in MS lifetimes due to ADM energy transport. \DIFdelbegin \DIFdel{\S~\ref{sub:isochrones} discusses }\DIFdelend \DIFaddbegin \DIFadd{We discuss }\DIFaddend the effects on surface properties of individual stars \DIFdelbegin \DIFdel{and }\DIFdelend \DIFaddbegin \DIFadd{in \S~\ref{sub:tracks}, and on }\DIFaddend the isochrones of stellar populations \DIFaddbegin \DIFadd{in \S~\ref{sub:isochrones}}\DIFaddend .
Note that all logarithms in this paper are base 10.



%------------------------------------------
\subsection{Low-Mass Stars}
\label{sub:lowmass}

  % 1.0Msun, energy and temperature
  \begin{figure}
    \centering
    \includegraphics[width=0.47\textwidth]{plots/m1p0.png}
    \caption{$1.0 \Msun$ profiles for \nodm (grey) and $\gbpow{6}$ (dark blue) models. Each set of 3 panels shows stellar profiles of the stars at different evolutionary phases indicated by the fraction of hydrogen in the center, $X_c$, which decreases as the star evolves \DIFaddbeginFL \DIFaddFL{(ZAMS is "zero-age main sequence"}\DIFaddendFL . The profiles in each panel are: 1) $\epsnuc$, the nuclear burning rate in [erg/g/s]; 2) $\epsx$, the rate at which DM transports energy (negative values indicate that energy is being removed), also in [erg/g/s]; 3) log($T$/K), log$_{10}$ of the temperature in [K]. ADM energy transport decreases the temperature and burning rate in the center and increases them in a shell at $m(<r) \sim 0.1 \Msun$. The $\gbpow{6}$ model reaches the $X_c$ evolutionary markers at older ages, relative to the \nodm model, because the changes to the burning rates cause the central hydrogen fraction to decline more slowly.
    }
    \label{fig:m1p0c6}
  \end{figure}


Standard model stars in the mass range \mrangelow have relatively 
low central temperatures and so are powered primarily by the pp chain, 
which is much less sensitive to the temperature than burning via the CNO cycle. 
This means the burning does not peak as strongly at the center and 
radiative transport is sufficient to carry the energy flux, so the core is radiative. 
Without the mixing provided by convection, 
hydrogen depletes first at the very center 
and the burning shifts gradually outward into a shell.


As seen in Figure~\ref{fig:m1p0c6}, 
energy transport by large amounts of ADM causes flatter temperature \DIFdelbegin \DIFdel{profiles 
}\DIFdelend \DIFaddbegin \DIFadd{gradients in the center 
}\DIFaddend than those seen in the \nodm model. This reduces the burning rate 
in the center (where ADM is removing energy) and increases 
it in a shell (where ADM deposits energy). Indeed, 
the increased burning at 
larger radii ($m(<r) \sim 0.1 \, \Msun$) 
causes a small net increase in the total 
luminosity of the star. Because ADM probes temperature differences 
over large portions of the star, 
energy transport by ADM remains efficient despite the shallower 
temperature gradient. The increased temperature at $m(<r) \gtrsim 0.1 \Msun$ 
means that more hydrogen burns during the MS in low-mass ADM models 
than in their \nodm counterparts. These two competing effects 
dictate the gross evolution of the star. For low values of 
$\gb$ (weaker ADM influence), the increase in the total 
amount of fuel wins and the stellar MS lifetime is an  
increasing function of $\gb$. At values of $\gb \gtrsim 10^3$, 
the burning rate continues to increase, \DIFaddbegin \DIFadd{but }\DIFaddend little additional 
fuel is burned because of the precipitous drop in temperature 
at larger radii, which cannot be overcome by the energy transport. 
The result is that the MS lifetime of the star decreases 
with increasing environmental factor for $\gb \gtrsim 10^3$. 
(We will discuss stellar lifetimes further in \S~\ref{sub:mstau}).
Stars in this regime have a higher surface luminosity than their \nodm counterparts, at fixed central hydrogen fraction\DIFdelbegin \DIFdel{, and move along their }\DIFdelend \DIFaddbegin \DIFadd{. However, at fixed luminosity we should expect their effective temperatures to be roughly the same, because these are equilibrium points dictated by the equations of stellar structure. Indeed, this is what we see in our models. The result is that stars of a given mass but different ADM content move along roughly the same }\DIFaddend tracks in an HR diagram\DIFdelbegin \DIFdel{at a faster pace}\DIFdelend \DIFaddbegin \DIFadd{, but they do so at different rates}\DIFaddend . (We will discuss stellar tracks \DIFaddbegin \DIFadd{in \S~\ref{sub:tracks} }\DIFaddend and population isochrones in \S~\ref{sub:isochrones}.)
% \arz{I would try to rephrase the following. This is not too shocking. For a 
% given luminosity, the structure is more or less set by the equations 
% of stellar structure. So we shouldn't expect a given luminosity to 
% correspond to a very different surface temperature under most 
% circumstances. However, what is different is the time sequence 
% of events. }


Previous work by \cite{Iocco+12} found that solar mass stars 
in dense ADM environments may experience dramatic oscillations 
in both their core structure and surface properties, 
due to the large amounts of energy that ADM removes from the center, 
but noted that they were unable to determine whether it was 
a physical effect or a numerical artifact. Our initial simulations, 
using \mesa release $10398$, resulted in similar oscillations. In those models, 
ADM energy transport caused the temperature profile 
to become inverted, with the central temperature falling below the ADM temperature. 
Once this occurred, ADM began transporting energy back toward the center of the star. 
This raised the central temperature until it surpassed the ADM temperature again, 
causing ADM energy transport to again reverse direction and move energy away from the center. 
This cycle was self-reinforcing, causing large oscillations in the core temperature, 
density, etc., which propagated outward and resulted in large 
oscillations in surface properties as well.


Upon further investigation, 
we found that these simulations had poor energy conservation 
due to numerical artifacts. 
A new \mesa version had been released since we had begun 
this work that included improved energy conservation schemes. 
When we updated to \mesa release $12115$ and ran these models again, 
the energy conservation was much improved, 
and the central temperature was reduced such that 
it was very close to the ADM temperature but never 
dropped below it. Since $\Tc > \Tx$ throughout the star's lifetime, 
ADM energy transport never reverses direction and the oscillations 
seen previously are absent. We conclude that the oscillations in our initial simulations
were a numerical artifact.


%------------------------------------------
\subsection{High-Mass Stars}
\label{sub:highmass}

  % 3.5Msun profiles, energy and convection
  \begin{figure}
    \centering
    \includegraphics[width=0.47\textwidth]{plots/m3p5.png}
    \caption{Same as Figure~\ref{fig:m1p0c6} for $3.5 \Msun$ models, except that the 3rd panel in each set shows log($D$), where $D$ is the diffusion coefficient for convective mixing in [cm$^2$/s]. In the \nodm model the convective core retreats \textit{toward} the center over time, and the burning rate peaks at the center until the end of the main sequence when the burning rate drops dramatically and a shell of strong burning appears suddenly. In the $\gbpow{6}$ model, convection at the very center shuts off relatively early in the MS and a convective shell retreats \textit{away} from the center over time. The peak burning rate shifts gradually outward, following the inner edge of the convective shell. The $\gbpow{6}$ model reaches the $X_c$ evolutionary markers at younger ages, relative to the \nodm model, since convection cannot replenish the fuel at the center.
    }
    \label{fig:m3p5}
  \end{figure}

In standard models, MS stars with $\Mstar \gtrsim 1.3 \Msun$ 
are powered primarily by the CNO cycle.
This has several important consequences:
(1) the burning rate is much higher than in pp-dominated stars;
(2) the burning rate is extremely sensitive to core temperature; 
(3) temperature gradients in the stellar core are relatively steep; 
and (4) stellar cores must be convective in order 
to carry away the energy produced by core hydrogen burning. 
Convective energy transport in the star also replenishes the 
core with unburnt hydrogen as the star evolves.
Once hydrogen throughout the convective zone is depleted, 
the burning rate rapidly decreases and 
the star loses more energy at its surface than is 
being generated by burning. Gravity temporarily overcomes pressure support and 
the star contracts until the internal temperature increase is sufficient 
to ignite hydrogen in a shell outside the depleted core. This restructuring 
produces the so-called "convective hook" in an HR diagram 
as the star leaves the main sequence \citep{Kippenhahn2012}.

If a star captures enough ADM, the combination of dark matter + radiative energy 
transport becomes sufficient to carry the flux from nuclear burning at a 
temperature gradient that is insufficient to support convection. 
In other words, the additional energy transport by ADM can turn off convection within the stellar core. 
This can be seen in 
Figure~\ref{fig:m3p5} for a $3.5 \Msun$, $\gbpow{6}$ star. 
Convection disappears from the center first 
(where ADM energy transport is most efficient) 
and retreats away from the core, into a narrowing shell. 
Without convective mixing, the hydrogen fuel supply depletes first 
at the very center (instead of simultaneously throughout the core) and 
the burning concomitantly shifts gradually into a shell, 
following the lower boundary of the convective zone. 
This can be seen in the time progression (down the page) 
of the $\gbpow{6}$ (dark blue) model in Figure~\ref{fig:m3p5}. 
The shift to shell burning is similar to the behavior of standard {\em low mass} 
stars.


%------------------------------------------
\DIFdelbegin \subsection{\DIFdel{Main Sequence Lifetimes}}
%DIFAUXCMD
\addtocounter{subsection}{-1}%DIFAUXCMD
%DIFDELCMD < \label{sub:mstau}
%DIFDELCMD < 

%DIFDELCMD < %%%
\DIFdelend % mstau plot
\begin{figure*}
  \centering
  \includegraphics[width=\textwidth]{plots/MStau.png}
  \caption{Changes in MS lifetimes, relative to a star of the same mass with no dark matter, seen in our simulations.
  Diamonds mark the transition from radiative to convective cores (left to right). For the purposes of this figure this is defined as the lowest $\Mstar$ for which the average (over the MS) mass of the convective core is greater than 0.01 Mstar. Stars to the right of the \nodm marker (grey diamond) have decreased lifetimes due to a reduction in the size of the convective core, which reduces the amount of hydrogen available for burning. The effect abruptly disappears as stellar lifetimes become shorter than the time required to build up a sufficient amount of ADM. Stars to the left of this marker show mixed behavior due to the competing effects of decreased central burning rates and higher temperatures around $m(<r) \gtrsim 0.1 \Msun$ which give the star access to more fuel. In addition to these trends, there are several abrupt dips (e.g., at $2.4 \Msun$) and spikes (e.g., at $2.55 \Msun$). This is due to rotational mixing that turns on \DIFdelbeginFL \DIFdelFL{in a shell }\DIFdelendFL part-way through the MS and funnels fresh hydrogen fuel to the center, which increases the lifetimes (spikes). The dips result when the \nodm model exhibits this feature, but the ADM model of the same mass does not.
  }
  \label{fig:mstau}
\end{figure*}

\DIFaddbegin \subsection{\DIFadd{Main Sequence Lifetimes}}
\label{sub:mstau}

\DIFaddend In Figure~\ref{fig:mstau} we summarize the effects of ADM on main sequence (MS) lifetimes relative to a standard \nodm star of the same mass. For the purposes of this \DIFdelbegin \DIFdel{plot}\DIFdelend \DIFaddbegin \DIFadd{paper}\DIFaddend , we have {\em defined} the MS to end when the fractional abundance of hydrogen in the center, $X_c$, falls below $10^{-3}$. \DIFdelbegin \DIFdel{$X_c$ declines rapidly }\DIFdelend \DIFaddbegin \DIFadd{Once $X_c < 10^{-3}$ the hydrogen burning rate is greatly reduced and the star transitions out of the MS and onto the sub-giant branch. This transition period is marked by relatively sudden and dramatic changes to the star's structure. Stars that capture large amounts of ADM can have significantly different core structures }\DIFaddend at the end of the MS \DIFdelbegin \DIFdel{, 
so our results are not strongly affected by the exact choice .
}\DIFdelend \DIFaddbegin \DIFadd{than their standard model counterparts, and these differences affect the stars' transition out of the MS, including the duration, in ways that are qualitatively different than ADM's effect on the MS itself. Therefore different choices in the definition of when a star leaves the MS can affect the results. Our relatively moderate choice of $10^{-3}$ highlights changes in the core of the star during the bulk of the MS, rather than changes in the transition period between the MS and the sub-giant branch. We discuss ADM's affects on this transition period in \S~\ref{sub:tracks}.
}\DIFaddend 

\DIFdelbegin \DIFdel{Lifetimes of low mass stars , which have radiative (not convective) cores even in }%DIFDELCMD < \nodm %%%
\DIFdel{models, 
}\DIFdelend \DIFaddbegin \DIFadd{The MS lifetimes of relatively 
low-mass stars (near $\sim 1 \Msun$) }\DIFaddend can be altered by up to 20\%\DIFdelbegin \DIFdel{. The }\DIFdelend \DIFaddbegin \DIFadd{, however the }\DIFaddend sense and degree of the shift \DIFdelbegin \DIFdel{in MS lifetime for relatively 
low-mass stars near $\Mstar \sim \Msun$ }\DIFdelend is not a monotonic function of the strength of the 
dark matter effect, parameterized by $\gb$. This complicated dependence on the amount of 
captured \DIFdelbegin \DIFdel{dark matter }\DIFdelend \DIFaddbegin \DIFadd{ADM }\DIFaddend is due to the competition between \DIFdelbegin \DIFdel{increases burning rate }\DIFdelend \DIFaddbegin \DIFadd{increased burning rates }\DIFaddend and increased 
availability of burnable hydrogen fuel as discussed in \S~\ref{sub:lowmass}. 


At higher masses, the influence of ADM on stellar lifetimes is clearer. 
ADM shortens the lifetimes of high mass stars ($\Mstar \gtrsim 1.3 \Msun$). 
In \nodm models, the central convection zone extends beyond the burning region, 
giving the star a source of fresh nuclear fuel as hydrogen from outside 
of the core is mixed into the center. 
Since ADM shuts off convection in the center, 
the star no longer gets this influx of fresh hydrogen. 
Consequently, the star has less fuel available to burn, 
and so it leaves the MS faster than the \nodm models. 
Note that the appearance of a convective core \DIFaddbegin \DIFadd{(diamonds, Fig~\ref{fig:mstau}) }\DIFaddend shifts 
to higher masses with increasing $\gb$ 
\DIFdelbegin \DIFdel{(diamonds in Figure~\ref{fig:mstau}) 
}\DIFdelend due to larger amounts of ADM which can carry larger energy fluxes. 
The effect disappears abruptly as $\Mstar$ increases because stellar 
lifetimes \DIFdelbegin \DIFdel{(which }\DIFdelend scale as $\Mstar^{\sim 2.5}$ \DIFdelbegin \DIFdel{) 
}\DIFdelend \DIFaddbegin \DIFadd{and 
}\DIFaddend quickly become too short for a sufficient amount of 
ADM to build up
\DIFdelbegin \DIFdel{in the star 
}\DIFdelend (recall that the ADM capture rate scales roughly linearly with $\Mstar$), 
while the luminosity of the star increases rapidly with mass 
(roughly, $L \propto \Mstar^{3.5}$), 
meaning that more energy must be transported 
in order to alter the stellar structure.

In addition to the trends we have discussed, Figure~\ref{fig:mstau} has several abrupt dips (e.g., at $2.4 \Msun$) and spikes (e.g., at $2.55 \Msun$). This is due to rotational mixing that turns on \DIFdelbegin \DIFdel{in a shell }\DIFdelend part-way through the MS and funnels fresh hydrogen \DIFdelbegin \DIFdel{fuel }\DIFdelend to the center, which increases the lifetimes (spikes). The dips result when the \nodm model exhibits this feature, but the ADM model of the same mass does not. This rotational mixing occurs sporadically (i.e. at isolated masses, not in a continuous range of masses) in our models and may or may not be physical. However, it cannot be a bug introduced by our ADM energy transport module since \DIFdelbegin \DIFdel{it happens in both ADM and }\DIFdelend \DIFaddbegin \DIFadd{some }\DIFaddend \nodm models \DIFdelbegin \DIFdel{. This type of behavior (stars of }\DIFdelend \DIFaddbegin \DIFadd{display the feature, but our module is not called in this case. This phenomenon where a star of }\DIFaddend a given mass \DIFdelbegin \DIFdel{experiencing increases in their central hydrogen fractions }\DIFdelend \DIFaddbegin \DIFadd{receives an influx of hydrogen to the center }\DIFaddend due to the onset of mixing, while stars of bracketing masses experience no such mixing\DIFdelbegin \DIFdel{) }\DIFdelend \DIFaddbegin \DIFadd{, }\DIFaddend has also been reported previously in the \mesa mailing lists\DIFaddbegin \footnote{\url{https://lists.mesastar.org/pipermail/mesa-users/}}\DIFaddend .


%------------------------------------------
\subsection{Stellar Evolutionary Tracks\DIFdelbegin \DIFdel{and Isochrones}\DIFdelend }
\DIFdelbegin %DIFDELCMD < \label{sub:isochrones}
%DIFDELCMD < %%%
\DIFdelend \DIFaddbegin \label{sub:tracks}
\DIFaddend 

% stellar tracks plots
\begin{figure*}
  \centering
  \includegraphics[width=\textwidth]{plots/tracks.png}
  \caption{
  Stellar evolution tracks, from ZAMS to core helium depletion (\DIFdelbeginFL \DIFdelFL{$Y_c<10^{-3}$}\DIFdelendFL \DIFaddbeginFL \DIFaddFL{$Y_c=10^{-3}$}\DIFaddendFL ), of select masses with $\gbpow{4}$ (left) and $\gbpow{6}$ (right). Tracks of \nodm models of the same mass are overplotted with a higher transparency. The chosen masses highlight some of the most dramatic changes seen in Figure~\ref{fig:mstau}. \DIFdelbeginFL \DIFdelFL{x's }\DIFdelendFL \DIFaddbeginFL \DIFaddFL{Triangles }\DIFaddendFL mark the location where stars leave the MS, which we define here as core hydrogen depletion below $X_c=10^{-3}$. The location of the ZAMS and core hydrogen depletion for \nodm models are plotted as dotted and solid black lines respectively. The spike in the $X_{c,\ NoDM} < 10^{-3}$ line near the $2.5 \Msun$ track is due to rotational mixing in the $2.4 \Msun$ model, which is discussed in \S~\ref{sub:mstau}. The main effect of ADM on a star's surface properties is to move the star through roughly the same sequence of events at a faster or slower pace, causing the offset of the \DIFdelbeginFL \DIFdelFL{$X_c<10^{-3}$ }\DIFdelendFL \DIFaddbeginFL \DIFaddFL{$X_c=10^{-3}$ }\DIFaddendFL milestone relative to \nodm. %DIF <    \arz{Need to beef up this caption a bit, ... e.g., thin vs. thick lines, values of stellar mass corresponding to each line, etc...}
  \DIFaddbeginFL \DIFaddFL{High mass stars with sufficient ADM skip the convective hook because ADM shuts off convection in the core. These stars transition into shell burning, and therefore the sub-giant branch, more smoothly, similar to low mass stars.
  }\DIFaddendFL }
  \label{fig:tracks}
\DIFaddbeginFL \end{figure*}
\DIFaddend 

%DIF >  Teff tracks
\DIFaddbegin \begin{figure*}
  \centering
  \includegraphics[width=\textwidth]{plots/Teff.png}
  \caption{
  \DIFaddFL{Same as Figure~\ref{fig:tracks} except that here we plot effective temperature as a function of stellar age. The sub-giant branch phase in a star's evolution is seen here as a sharp drop in $\Teff$, and is a result of structural changes in the star that are triggered by large reductions in the core burning rate due to hydrogen depletion at the end of the MS. The duration of the transition period between the MS and the sub-giant branch is seen here as the temporal difference between the locations of $X_c = 10^{-3}$ (triangles) and the drop in $\Teff$. ADM alters the duration of the transition period, tending to increase it in high mass stars and decrease it in low mass stars. This is opposite of ADM's effect on MS lifetimes. The net effect is that the feature in $\Teff$ always occurs either concurrently or earlier in the ADM model than in its }\nodm \DIFaddFL{counterpart.
  }}
  \label{fig:teff}
\DIFaddendFL \end{figure*}
\DIFaddbegin 

\DIFaddend %==============================================================


% \arz{I think we can work under the assumption that the reader 
% has at least passing familiarity with the HR diagram, the 
% ``theorists'' HR diagram (the T-L plane), and some very basics of 
% stellar evolution. So, I might rephrase as 
% follows. $\rightarrow$} 
One of the goals of this work is to determine whether or 
not ADM can cause any gross changes to the properties of 
stars. We begin to answer this question with 
Figure~\ref{fig:tracks}, 
which shows evolutionary tracks on the 
HR diagram for many of our models. The tracks begin on the 
zero-age main sequence (ZAMS), delineated by the dotted 
black lines at the lower left of each panel. 
Stars evolve off of the ZAMS in a mass-dependent manner that is 
familiar from well-known aspects of standard stellar evolution. 
The tracks that we show leave the MS, defined as $X_{\rm c} < 10^{-3}$, 
at the points marked by \DIFdelbegin \DIFdel{crosses}\DIFdelend \DIFaddbegin \DIFadd{triangles}\DIFaddend . The points at which stars exit  
the MS in a standard model with no dark matter are indicated by 
the solid black lines in each panel. \DIFaddbegin \DIFadd{(The spike near the $2.5 \Msun$ track is due to rotational mixing in the $2.4 \Msun$ model, which is discussed in \S~\ref{sub:mstau}.) }\DIFaddend Stars spend the majority of 
their lives on the MS and move more rapidly through the subsequent 
phases of stellar evolution. Our evolutionary tracks terminate 
when the core helium fraction falls below $10^{-3}$.
% \arz{$\leftarrow$ End rephrase. If you like that rephrase, then 
% we can get rid of the remainder of this paragraph. $\rightarrow$}
% The HR diagrams in Figure~\ref{fig:tracks} show the effects of ADM 
% on the luminosity and temperature 
% at the stellar surface from the zero age main sequence 
% (“ZAMS”, bottom of each track) through helium core burning. 
% $L$ and $\Teff$ are closely related to the star's magnitude 
% and color which are observable properties. Stars spend most 
% of their time on the MS, burning hydrogen in their cores, 
% and evolve up the track more quickly after leaving the MS.
% \arz{$\leftarrow$}


As is evident in Fig.~\ref{fig:tracks}, 
the effects of ADM on the evolutionary track 
of any individual star are generally subtle. 
Roughly speaking, this is not surprising. 
\DIFdelbegin \DIFdel{The cooling induced by ADM effects only stellar 
cores. The chemical compositions, }\DIFdelend %DIF >  The cooling induced by ADM effects only stellar cores. 
\DIFaddbegin \DIFadd{At a fixed central hydrogen fraction, stars containing ADM have different surface luminosities than their standard model counterparts, which is a result of ADM altering the structure of the star. This can be seen in the difference in location between the $X_c=10^{-3}$ markers in Fig.~\ref{fig:tracks} (x's for ADM models, solid black line for }\nodm \DIFadd{models).
However, if we instead consider stars at fixed luminosity, the }\DIFaddend temperature profiles, \DIFaddbegin \DIFadd{chemical compositions, 
}\DIFaddend opacities, and other properties of the overlying zones 
are approximately unaltered by ADM cooling. Consequently, 
the gross properties of the stellar photosphere, which are 
determined via the equations of stellar structure, are 
approximately fixed\DIFaddbegin \DIFadd{, }\DIFaddend at fixed luminosity. 
\DIFaddbegin \DIFadd{The result is that both ADM and standard model stars of a given mass follow roughly the same tracks in the HR diagram, but they do so at different rates.
}\DIFaddend 


Nonetheless, there are some small differences between 
standard evolutionary tracks and the tracks of stars 
with ADM. For example, consider the track of the 
$1.75 \Msun$ star in the left panel, corresponding 
to an environmental factor of $\gb = 10^4$.
The standard 
model of stellar evolution shows a kink in the evolutionary 
track as the star exits the MS. This kink is known as the 
{\em convective hook}. The convective hook is caused by 
an overall contraction of the convective cores of the 
stars after hydrogen depletion. During this phase, 
$T_{\rm eff}$ increases. Eventually, at the hottest point 
on the hook, contraction of the former convective core 
is sufficient to ignite burning in a shell. After this 
point, shell burning ensues and the star continues to 
evolve along the sub-giant branch. What is clear from 
the evolutionary track of the $1.75M_{\odot}$ star 
in the left panel of Fig.~\ref{fig:tracks} is that 
this evolutionary track exhibits {\em no convective hook.} 
This is because convection within the stellar core has 
been shut off by the ADM in these models. Instead of 
going through a phase of core collapse followed by shell 
burning, such stars make a smooth transition to shell burning 
and, thus, a smooth transition to the sub-giant branch. 
The absence of convective hooks is evident 
for a wider range of masses in the right hand 
panel of Fig.~\ref{fig:tracks}, which corresponds 
to a larger environmental factor of $\gb=10^6$. 


The convective hook feature has been clearly 
seen in many open clusters for which 
the main sequence turn off lies between $\sim 1.3\Msun$ 
and $\sim 2\Msun$, corresponding to stellar 
ages of $\sim 1\, \mathrm{Gyr}$ to $\sim 4\mathrm{Gyr}$. 
However, this does not yet provide any strong 
statement about the nature of dark matter because 
none of those environments are thought to contain 
significant amounts of dark matter. 
Exploiting the convective hook feature 
would require the serendipitous identification 
of a stellar population of the appropriate age, associated with a significant amount of dark matter, and containing a sufficiently
large number of stars such that the ephemeral 
convective hook and sub-giant phases of 
evolution are well sampled.



%DIF <  \arz{I rewrote a lot of what is above. I think the 
%DIF <  stuff below here can be removed if you agree $\rightarrow$}
%DIF <  since the most significant effect on the surface properties 
%DIF <  is to move the star through roughly the same sequence 
%DIF <  of events at a more rapid pace. The location of the ZAMS 
%DIF <  is set by the balance between pressure support from nuclear 
%DIF <  burning and gravitational contraction, which depends only on the mass. 
%DIF <  The temperature of the red giant branch (vertical portion of the tracks) 
%DIF <  is set by an equilibrium between neutral and ionized hydrogen 
%DIF <  in the atmosphere. Hence, the star's HR track cannot be too different from the \nodm models.
%DIF <  \arz{$\leftarrow$.}
\DIFaddbegin \DIFadd{In Figure~\ref{fig:teff} we plot the effective temperatures with respect to stellar age to better understand the transition from the MS to the sub-giant branch, seen here as the difference between $X_c = 10^{-3}$ (triangles) and the sharp drop in $\Teff$. A star's exit off of the main sequence is triggered when the core hydrogen fuel supply is depleted and the burning rate decreases such that it no longer provides sufficient pressure support and the star begins to collapse. Densities and temperatures increase until the bottom layer of hydrogen, now in a shell surrounding the core, ignites. The outward pressure resulting from increased shell burning causes the star's outer envelope to expand and cool at roughly constant luminosity, seen in Figure~\ref{fig:teff} as a large, sudden drop in $\Teff$. In models with no ADM, this transition period is more abrupt in high mass stars due to the mixing induced by their convective cores, which causes hydrogen to become depleted throughout the core simultaneously and shell burning to appear suddenly (see \S~\ref{sub:highmass}). In standard model low mass stars the cores are not convective, and so hydrogen depletes first at the very center and the burning shifts into a shell more gradually (see \S~\ref{sub:lowmass}). 
}\DIFaddend 

\DIFaddbegin \DIFadd{The temporal difference between ADM and standard models in the location of the sudden drop in $\Teff$ (when increased shell burning causes the envelope to expand) is another indicator of the change in MS lifetime. Unlike our definition of the end of the main sequence ($X_c = 10^{-3}$) this indicator is based on surface properties and occurs towards the end of the large structural changes the happen during the transition period. In some cases (e.g., in the $1.75 \Msun$, $\gbpow{4}$ model) this temporal difference is much smaller than the change in MS lifetime given by our definition of leaving the MS (seen here as the difference between the triangle markers of the ADM model and its standard model counterpart), and in other cases (e.g., in the $1.0 \Msun$, $\gbpow{6}$ model) it is larger.
}

\DIFadd{ADM can affect both burning rates and stellar structure (e.g., convection), and therefore it is not surprising that ADM affects the timescale of a star's transition off of the main sequence. High mass stars that skip the convective hook due to ADM energy transport take longer to move through this transition period because the burning shifts gradually into a shell (see \S~\ref{sub:highmass} for details). This behavior is very similar to standard model low mass stars. Conversely, the $1.0 \Msun$ AMD models move through this period more quickly than their }\nodm \DIFadd{counterpart. This is likely due to the fact that ADM has caused higher burning rates at the outer edge of the core during the MS, so that mixing during this transition period brings more helium into the center than in the standard model star. These shifts are opposite the shifts in MS lifetimes, and the net result is that this feature in $\Teff$ always occurs either concurrently or earlier in the ADM model than in its standard model counterpart. 
}


%DIF > ------------------------------------------
\subsection{\DIFadd{Stellar Population Isochrones}}
\label{sub:isochrones}


\DIFaddend % isochrones plots
\begin{figure*}
\centering
   \DIFdelbeginFL %DIFDELCMD < \includegraphics[width=\textwidth]{plots/isos_cb4.png}
%DIFDELCMD <   %%%
\DIFdelendFL \DIFaddbeginFL \begin{subfigure}{0.49\linewidth} \centering
     \includegraphics[scale=0.4]{plots/isos_cb4.png}
    %DIF >   \caption{Figure on left side}
    %DIF >   \label{fig:figA}
   \end{subfigure}
   \begin{subfigure}{0.49\linewidth} \centering
     \includegraphics[scale=0.4]{plots/isos_cb6.png}
    %DIF >   \caption{Figure on right side}
    %DIF >   \label{fig:figB}
   \end{subfigure}
\DIFaddendFL \caption{\DIFaddbeginFL \DIFaddFL{Isochrones for }\DIFaddendFL $\gbpow{4}$ \DIFdelbeginFL \DIFdelFL{isochrones }\DIFdelendFL \DIFaddbeginFL \DIFaddFL{(left) and $\gbpow{6}$ (right) models, marked by circles, }\DIFaddendFL with \nodm models overplotted \DIFdelbeginFL \DIFdelFL{as thin lines}\DIFdelendFL \DIFaddbeginFL \DIFaddFL{at higher transparency and marked by crosses}\DIFaddendFL . \DIFdelbeginFL \DIFdelFL{Triangles mark $3.5 \Msun$, circles mark $1.0 \Msun$. }\DIFdelendFL The \DIFdelbeginFL \DIFdelFL{positions of the markers here }\DIFdelendFL \DIFaddbeginFL \DIFaddFL{data points }\DIFaddendFL are \DIFdelbeginFL \DIFdelFL{very similar to }\DIFdelendFL \DIFaddbeginFL \DIFaddFL{interpolated from }\DIFaddendFL the \DIFdelbeginFL %DIFDELCMD < \nodm %%%
\DIFdelFL{case }\DIFdelendFL \DIFaddbeginFL \DIFaddFL{stellar tracks to a common set of times }\DIFaddendFL (\DIFdelbeginFL \DIFdelFL{not shown}\DIFdelendFL \DIFaddbeginFL \DIFaddFL{isochrone ages}\DIFaddendFL ). \DIFdelbeginFL \DIFdelFL{Gaps in }\DIFdelendFL \DIFaddbeginFL \DIFaddFL{We connect }\DIFaddendFL the \DIFaddbeginFL \DIFaddFL{interpolated }\DIFaddendFL data \DIFaddbeginFL \DIFaddFL{points of a single isochrone with straight lines to guide the eye. The lowest data point on every isochrone is a $0.9 \Msun$ star. As stars leave the main sequence they evolve rapidly, and therefore subsequent phases }\DIFaddendFL are \DIFaddbeginFL \DIFaddFL{less well sampled }\DIFaddendFL due to the \DIFdelbeginFL \DIFdelFL{difficulty interpolating in regions where }\DIFdelendFL \DIFaddbeginFL \DIFaddFL{mass resolution. Isochrone ages have been chosen to maximize }\DIFaddendFL the \DIFdelbeginFL \DIFdelFL{evolution }\DIFdelendFL \DIFaddbeginFL \DIFaddFL{sampling around the MS turnoff and sub-giant branch. We do not show the giant branches because we do not have enough data points there to be meaningful.
  The MS turnoff }\DIFaddendFL of \DIFaddbeginFL \DIFaddFL{isochrones around 1 Gyr happens at a higher effective temperature and a }\DIFaddendFL lower \DIFaddbeginFL \DIFaddFL{luminosity and skips the convective hook. This happens in a wider range of ages in populations that live in richer ADM environments (right panel). The oldest isochrones contain only low }\DIFaddendFL mass stars\DIFdelbeginFL \DIFdelFL{is }\textit{\DIFdelFL{faster}} %DIFAUXCMD
\DIFdelFL{than }\DIFdelendFL \DIFaddbeginFL \DIFaddFL{, and the ADM and }\nodm \DIFaddFL{populations look very similar except }\DIFaddendFL that \DIFdelbeginFL \DIFdelFL{of stars with }\DIFdelendFL \DIFaddbeginFL \DIFaddFL{populations in high $\gb$ environments appear }\DIFaddendFL slightly \DIFdelbeginFL \DIFdelFL{higher masses (see \S~\ref{sub:isochrones})}\DIFdelendFL \DIFaddbeginFL \DIFaddFL{older due to their decreased luminosity and temperature}\DIFaddendFL .
  } 
\DIFdelbeginFL %DIFDELCMD < \label{fig:isos_cb4}
%DIFDELCMD < %%%
\DIFdelendFL \DIFaddbeginFL \label{fig:isos}
\DIFaddendFL \end{figure*}


%DIF >  MS turnoff (hottest MS star) Teff and L plots
\begin{figure*}
    \centering
    \DIFdelbeginFL %DIFDELCMD < \includegraphics[width=\textwidth]{plots/isos_cb6.png}
%DIFDELCMD <   %%%
\DIFdelendFL \DIFaddbeginFL \includegraphics[width=\textwidth]{plots/hotTeff.png}
    \DIFaddendFL \caption{\DIFdelbeginFL \DIFdelFL{Same }\DIFdelendFL \DIFaddbeginFL \DIFaddFL{Effective temperature of the main-sequence-turnoff star, defined here }\DIFaddendFL as \DIFdelbeginFL \DIFdelFL{Figure~\ref{fig:isos_cb4} but for }\DIFdelendFL \DIFaddbeginFL \DIFaddFL{the hottest MS star, at a given age.
    We show only the }\nodm\DIFaddFL{, $\gbpow{4}$, and }\DIFaddendFL $\gbpow{6}$ \DIFaddbeginFL \DIFaddFL{models to allow the reader to see them clearly}\DIFaddendFL . The MS turnoff \DIFaddbeginFL \DIFaddFL{temperature }\DIFaddendFL of \DIFaddbeginFL \DIFaddFL{the $\gbpow{4}$ models starts to become hotter than in the }\nodm \DIFaddendFL isochrones \DIFdelbeginFL \DIFdelFL{between (insert ages) happens }\DIFdelendFL at \DIFdelbeginFL \DIFdelFL{a lower luminosity and skips }\DIFdelendFL \DIFaddbeginFL \DIFaddFL{$\sim 0.7$ Gyr when stars of $\sim 2 \Msun$ begin leaving }\DIFaddendFL the \DIFdelbeginFL \DIFdelFL{convective hook}\DIFdelendFL \DIFaddbeginFL \DIFaddFL{MS}\DIFaddendFL , \DIFdelbeginFL \DIFdelFL{causing them }\DIFdelendFL \DIFaddbeginFL \DIFaddFL{while those of the $\gbpow{6}$ models become hotter at $\sim 0.1$ Gyr when stars of $\sim 3.5 \Msun$ begin leaving the MS. As we move }\DIFaddendFL to \DIFdelbeginFL \DIFdelFL{appear }\DIFdelendFL older \DIFdelbeginFL \DIFdelFL{relative to }%DIFDELCMD < \nodm %%%
\DIFdelendFL isochrones \DIFaddbeginFL \DIFaddFL{the effect is reversed, and ADM models have lower effective temperatures at the turnoff}\DIFaddendFL . \DIFaddbeginFL \DIFaddFL{The lines terminate when there are no more stars on the main sequence (the lowest mass in our set of models is $0.9 \Msun$).
    For the purposes of this figure, we exclude stars in the convective hook because this feature is not well-resolved in our isochrones (see Fig.~\ref{fig:isos}).
    }\DIFaddendFL }
    \DIFdelbeginFL %DIFDELCMD < \label{fig:isos_cb6}
%DIFDELCMD < %%%
\DIFdelendFL \DIFaddbeginFL \label{fig:hotTeff}

\DIFaddendFL \end{figure*}

%DIF >  Isos from Dotter code
\DIFaddbegin \begin{figure}
    \centering
    \includegraphics[width=0.45\textwidth]{plots/iso_Dotter.png}
    \caption{\DIFaddFL{Isochrones generated by the }\mist \DIFaddFL{code for }\nodm \DIFaddFL{(grey) and $\gbpow{6}$ (dark blue) populations. The lowest mass star in both isochrones has $\Mstar = 2.25 \Msun$ (the interpolation was not successful for lower masses), and we show data through core helium depletion ($Y_c = 10^{-3}$) which corresponds to $\Mstar = 3.2 \Msun$ in both cases.
    The $\gbpow{6}$ isochrone skips the convective hook and crosses the sub-giant branch at a lower luminosity, consistent with Figure~\ref{fig:isos}. Additionally, the tip of its red giant branch occurs at a lower luminosity, a trend which appears in most of the $\gbpow{6}$ }\mist \DIFaddFL{isochrones with 0.1 Gyr $\lesssim$ Isochrone age $\lesssim$ 1 Gyr.}}
    \label{fig:isoDot}
\end{figure}


\DIFaddend Though the evolutionary tracks are quite similar across all models, 
ADM changes the rate of evolution and, hence, the stellar ages at 
which stars reach particular evolutionary stages. To convey some 
of the information that is obscured in an evolutionary track, 
we present stellar isochrones in Figure~\DIFdelbegin \DIFdel{\ref{fig:isos_cb4} and 
Figure~\ref{fig:isos_cb6}}\DIFdelend \DIFaddbegin \DIFadd{\ref{fig:isos}}\DIFaddend . Each isochrone is a line on this plot 
that represents the locus of points that would be occupied by a 
population of stars of fixed age, but a wide range of masses. 

%DIF < 
%DIF < The isochrones in Figures~\ref{fig:isos_cb4} and \ref{fig:isos_cb6} show the effects of ADM on a stellar 
%DIF < population. Each isochrone represents a group of stars at a fixed age with a range of masses. Stellar mass 
%DIF < increases from bottom to top.
%DIF <  , and the locations of $1 \Msun$ (circles) and $3.5 \Msun$ (triangles) stars are shown for reference.
%DIF < 
\DIFdelbegin %DIFDELCMD < 

%DIFDELCMD < \arz{Let's revisit these three paragraphs after we have a final decision about how we 
%DIFDELCMD < plan to present the isochrones. Ideally, it would be good to somehow better resolve the 
%DIFDELCMD < sub-giant and RG branches.}
%DIFDELCMD < %%%
\DIFdel{The change in MS lifetimes due to ADM is seen here }\DIFdelend \DIFaddbegin \DIFadd{The changes caused by ADM to individual stars' MS evolution is seen in these populations }\DIFaddend as a shift in the location of the MS turn-off\DIFdelbegin \DIFdel{(}\DIFdelend \DIFaddbegin \DIFadd{, }\DIFaddend where the isochrones take a hard right turn\DIFdelbegin \DIFdel{, e. g. , near the yellow triangle in Figure~\ref{fig:isos_cb6}). Because the general effect on stars that are altered by ADM is to shorten MSlifetimes, }\DIFdelend \DIFaddbegin \DIFadd{. We have chosen these particular ages to maximize the sampling around this period and the subsequent crossing of the sub-giant branch. We do not show the giant branches because we do not have enough data points there to be meaningful, but we discuss the red giant branch further below. Due to the fact that stars move through the MS turnoff, sub-giant, and giant branches rather quickly, our mass sampling limits our ability to resolve these phases.
}

\DIFadd{At around 1 Gyr, stars of $\Mstar \sim 1.75 \Msun$ are leaving the MS. In Figure~\ref{fig:isos}, we see that the MS turnoffs around this time occur at a higher effective temperature and skip the convective hook, reflecting ADM's affects on these stars, discussed in \S~\ref{sub:highmass}. These isochrones also tend to cross }\DIFaddend the \DIFdelbegin \DIFdel{turn-off happens at a lower mass and the stars evolve through the }\DIFdelend sub-giant branch at a lower luminosity. This \DIFdelbegin \DIFdel{causes isochrones of clusters in high $\gb$ environments to appear older than their standard model counterparts. This becomes noticeable in the highest $\gb$ environments around 0.2 Gyr as stars with $\rm{M} \approx 4 \Msun$ begin to leave the MS. Prior to this , the stars have not had enough time to capture a sufficient number of DM particles for ADM-driven energy transport to be a significant contribution to the overall energy transfer within the stars}\DIFdelend \DIFaddbegin \DIFadd{reflects the fact that AMD speeds up the evolution of these stars, and so stars with smaller initial masses, which have lower effective temperatures, are crossing the sub-giant branch earlier than they otherwise would. This happens in a wider range of ages in populations that live in richer ADM environments (right panel). Unfortunately stars move through this phase very quickly, meaning isochrones of real stellar populations are very sparsely populated in this region, which is known as the Hertzsprung Gap}\DIFaddend .

\DIFdelbegin \DIFdel{Younger }%DIFDELCMD < \nodm %%%
\DIFdel{isochrones display a sharp feature called the convective hook near the MS turn-off, which is noticeably absent from the $\gbpow{6}$ isochrones. Since the }%DIFDELCMD < \nodm %%%
\DIFdel{models have convective cores, hydrogen is depleted nearly simultaneously throughout the volume of the core at the end of the MS. This causes }\DIFdelend \DIFaddbegin \DIFadd{The oldest isochrones contain only low mass stars, since higher mass stars have already evolved into the giant branches and beyond. Here, }\DIFaddend the \DIFdelbegin \DIFdel{burning rate to decrease dramatically, which reduces the pressure support and results in gravitational contraction.
The contraction increases the temperature until the bottom layer of hydrogen (now in a shell surrounding the core) is hot enough to ignite, reestablishing the pressure support. Our high }\DIFdelend \DIFaddbegin \DIFadd{ADM isochrones look very similar to their standard model counterparts, except that populations in high }\DIFaddend $\gb$ \DIFdelbegin \DIFdel{isochrones lack this convective hook because ADM shuts off core convection. This lack of mixing causes the hydrogen to deplete first at the very center, and the burning gradually shifts into a shell, avoiding the sudden instability that causes the convective hook. This gradual shift to shell burning is very similar to the low mass }%DIFDELCMD < \nodm %%%
\DIFdel{models (which also do not experience a convective hook), which contributes to }\DIFdelend \DIFaddbegin \DIFadd{environments appear slightly older due to their decreased luminosity and temperature. This indicates that ADM causes the stars' surface properties to evolve more quickly, likely due to the increase in shell burning.
}

\DIFadd{To better resolve the isochrone's MS turnoffs, Figure~\ref{fig:hotTeff} shows the effective temperature of }\DIFaddend the \DIFdelbegin \DIFdel{high }\DIFdelend \DIFaddbegin \DIFadd{MS turnoff star, which we define as the hottest MS star at a given age. At younger ages there is no difference between the ADM and standard models because the stars have not yet captured enough ADM to be significantly affected. Around 0.15 Gyr, isochrones of the $\gbpow{6}$ model start to display higher temperatures, remaining high until $\sim 3$ Gyr, after which their temperatures are cooler than their }\nodm \DIFadd{counterparts. The $\gbpow{4}$ isochrones show similar trends, but they occur at later times, since it takes longer for stars in lower }\DIFaddend $\gb$ \DIFdelbegin \DIFdel{appearing older }\DIFdelend \DIFaddbegin \DIFadd{environments to build up sufficient ADM. The waviness in the lines at older ages is a result of limited mass resolution}\DIFaddend .

The \DIFdelbegin \DIFdel{gaps in the data are due to difficulty interpolating masses at a fixed time. Stellar masses generally increase when following a single isochrone from bottom to top. To build the isochrones from MESA's stellar models, MIST identifies the ages at which each star reaches a series of evolutionary milestones, and interpolates the stellar mass to generate properties of a group of stars at a fixed age. More massive stars generally evolve more quickly, and so the mapping between age and mass at a fixed milestone is usually monotonic, 
which the interpolation relies on. Exceptions to the monotonicity occur in mass regions where slightly less massive stars evolve more quickly. 
For example, in }\DIFdelend \DIFaddbegin \DIFadd{trends seen in our isochrones are consistent with what we were able to see in the isochrones generated by the }\mist \DIFadd{code, in regions where that code's interpolation was successful. In addition, we noticed from those isochrones that the tip of the red giant branch tends to occur at a lower luminosity in ADM models. The tip of the red giant branch is commonly used as a distance indicator, 
%DIF >  and is also known to occur at lower effective temperatures in more metal-rich populations, 
so ADM may add a source of uncertainty to these studies. 
To give the reader a sense of this shift, and to show a more well resolved MS turnoff and sub-giant branch, we show one particularly successful }\mist \DIFadd{isochrone for }\DIFaddend \nodm \DIFdelbegin \DIFdel{models, the emergence of a convective core around $1.3 \Msun$ results in stars just above this threshold living slightly }\textit{\DIFdel{longer}} %DIFAUXCMD
\DIFdel{than stars just below it. With large amounts of ADM, the increase in stellar lifetimes through this transition to a convective core is much larger than in }%DIFDELCMD < \nodm %%%
\DIFdel{models and occurs over a wider mass range. These regions are not interpolated and appear as "missing" data in the isochrones}\DIFdelend \DIFaddbegin \DIFadd{and $\gbpow{6}$ models in Figure~\ref{fig:isoDot}. The lowest mass star in both isochrones (lower left) has $\Mstar = 2.25 \Msun$ (the interpolation was not successful for lower masses), and we show data through core helium depletion ($Y_c = 10^{-3}$) which corresponds to $\Mstar = 3.2 \Msun$ in both cases}\DIFaddend .


%DIF <  MS turnoff (hottest MS star) Teff and L plots
\DIFdelbegin %DIFDELCMD < \begin{figure*}
%DIFDELCMD <     \centering
%DIFDELCMD <     \includegraphics[width=\textwidth]{plots/hotTeff.png}
%DIFDELCMD <     %%%
%DIFDELCMD < \caption{%
{%DIFAUXCMD
\DIFdelFL{PLACEHOLDER for MS turn off plots, Teff and/or L.
    Main sequence turnoff temperature (temperature of the hottest star that is still on the main sequence). Residuals are with respect to (...?)
    }%DIFDELCMD < \arz{Given a set of isochrones, isn't this plot somewhat redundant?}
%DIFDELCMD <     %%%
}
    %DIFAUXCMD
%DIFDELCMD < \label{fig:hotTeff}
%DIFDELCMD < %%%
\DIFdelendFL %DIF >  Because the general effect on stars that are altered by ADM is to shorten MS lifetimes, the turn-off happens at a lower mass and the stars evolve through the sub-giant branch at a lower luminosity. This causes isochrones of clusters in high $\gb$ environments to appear older than their standard model counterparts. This becomes noticeable in the highest $\gb$ environments around 0.2 Gyr as stars with $\rm{M} \approx 4 \Msun$ begin to leave the MS. Prior to this, the stars have not had enough time to capture a sufficient number of DM particles for ADM-driven energy transport to be a significant contribution to the overall energy transfer within the stars.

\DIFdelbeginFL %DIFDELCMD < \end{figure*}
%DIFDELCMD < %%%
\DIFdelend %DIF >  Younger \nodm isochrones display a sharp feature called the convective hook near the MS turn-off, which is noticeably absent from the $\gbpow{6}$ isochrones. Since the \nodm models have convective cores, hydrogen is depleted nearly simultaneously throughout the volume of the core at the end of the MS. This causes the burning rate to decrease dramatically, which reduces the pressure support and results in gravitational contraction. The contraction increases the temperature until the bottom layer of hydrogen (now in a shell surrounding the core) is hot enough to ignite, reestablishing the pressure support. Our high $\gb$ isochrones lack this convective hook because ADM shuts off core convection. This lack of mixing causes the hydrogen to deplete first at the very center, and the burning gradually shifts into a shell, avoiding the sudden instability that causes the convective hook. This gradual shift to shell burning is very similar to the low mass \nodm models (which also do not experience a convective hook), which contributes to the high $\gb$ appearing older.


\DIFdelbegin %DIFDELCMD < [%%%
\DIFdel{Paragraph(s) on MS turnoff plots (Teff and/or L as fnc of isochrone age)}%DIFDELCMD < ]
%DIFDELCMD < %%%
\DIFdelend %DIF >  The gaps in the data are due to difficulty interpolating masses at a fixed time. Stellar masses generally increase when following a single isochrone from bottom to top. To build the isochrones from MESA's stellar models, MIST identifies the ages at which each star reaches a series of evolutionary milestones, and interpolates the stellar mass to generate properties of a group of stars at a fixed age. More massive stars generally evolve more quickly, and so the mapping between age and mass at a fixed milestone is usually monotonic, which the interpolation relies on. Exceptions to the monotonicity occur in mass regions where slightly less massive stars evolve more quickly. For example, in \nodm models, the emergence of a convective core around $1.3 \Msun$ results in stars just above this threshold living slightly \textit{longer} than stars just below it. With large amounts of ADM, the increase in stellar lifetimes through this transition to a convective core is much larger than in \nodm models and occurs over a wider mass range. These regions are not interpolated and appear as "missing" data in the isochrones.


%------------------------------------------
\section{Discussion and Conclusions}
\label{sec:discus}

\DIFdelbegin %DIFDELCMD < \arz{I think that most of your conclusions section could be 
%DIFDELCMD < (1) a summary of our results and 
%DIFDELCMD < (2) speculations on how to turn this into a constraint in the future.}
%DIFDELCMD < %%%
\DIFdelend %DIF >  \arz{I think that most of your conclusions section could be 
%DIF >  (1) a summary of our results and 
%DIF >  (2) speculations on how to turn this into a constraint in the future.}

We have studied the potential impact of asymmetric dark matter interacting 
with nucleons through a spin-dependent coupling on the gross evolution 
of stars. We accomplished this by incorporating a module that approximates 
heat transport by dark matter into the {\tt MESA} stellar evolution 
software. We have identified several interesting qualitative distinctions between the standard evolution of stars and the evolution of stars in 
environments with a very high dark matter content. These include,
\DIFdelbegin %DIFDELCMD < \arz{List them.}
%DIFDELCMD < %%%
\DIFdelend \DIFaddbegin \begin{inparaenum}[1)]
\item \DIFadd{changes to the core temperature gradient, which alters the burning rates and stellar structures of low and high mass stars in qualitatively different ways;
}\item \DIFadd{low mass stars' MS lifetimes (defined as $X_c > 10^{-3}$) are }\emph{\DIFadd{increased}} \DIFadd{by as much as 20\%;
}\item \DIFadd{high mass stars' MS lifetimes are }\emph{\DIFadd{reduced}} \DIFadd{by as much as 40\%;
}\item \DIFadd{stars in both mass regimes cross the sub-giant branch at younger ages and may reach the tip of the red giant branch at lower luminosities.
}\end{inparaenum}
\DIFaddend 

It is interesting to speculate on ways in which these effects could be 
used to identify and/or constrain dark matter or ways in which these 
effects may, at least, serve as an element of uncertainty in the analysis 
of stellar populations. Any constraint on dark matter arising from these 
effects requires very high-quality observations of a stellar population 
residing in an environment with a large ambient dark matter density and thus there will be a significant element of serendipity involved. 
\DIFdelbegin %DIFDELCMD < \arz{List consequences.}
%DIFDELCMD < %%%
\DIFdelend \DIFaddbegin \DIFadd{If such a population is observed, our models suggest that if it is $\sim 0.1$ Gyr old, the hottest MS star should be slightly }\emph{\DIFadd{hotter}} \DIFadd{than expected for a population without ADM, and it should be slightly }\emph{\DIFadd{cooler}} \DIFadd{at $\sim 1$ Gyr. In addition, the tip of the red giant branch may occur at a lower luminosity. The tip of the red giant branch is commonly used as a distance indicator
%DIF >  , and is also known to occur at lower effective temperatures in more metal-rich populations, 
so ADM may add a source of uncertainty to these studies.
Finally, the metallicity is known to affect many of the properties we have discussed (e.g., the locations of various phases in the HR diagram), so ADM may be a contaminant here as well.
}\DIFaddend 

Future work along these lines includes 
\DIFdelbegin %DIFDELCMD < \arz{at minimum, it would 
%DIFDELCMD < to explore SI scattering.}
%DIFDELCMD < %%%
\DIFdelend \DIFaddbegin \begin{inparaenum}[1)]
\item \DIFadd{chemical abundance studies exploring the effects of altered core burning;
}\item \DIFadd{asteroseismology of Sun-like MS and red giant branch stars, which could be seen in (e.g.,) the small frequency separation--a diagnostic that is sensitive to the core structure of the star.
}\item \DIFadd{spin-independent ADM-nucleus scattering, which should have a larger effect during later phases when stars are burning helium.
}\end{inparaenum}
\DIFaddend 

%DIF >  {\color{red}TJH: One avenue for turning our results into a constraint is to explore the effect on asteroseismology. That's what I'm working on now. I'm optimistic to have a completed draft in a couple months.
\DIFaddbegin 

%DIF >  For future work we could say something like, ``ADM mainly effects the core of the star. It is possible that this could have observable effects on the asteroseismology of Sun-like Main Sequence and Red Giant Branch stars. This could be seen in e.g. the small frequency separation--a diganostic that is sensitive to the core structure of the star."}



\DIFaddend %------------------------------------------
\section*{Acknowledgements}



%%%%%%%%%%%%%%%%%%%% REFERENCES %%%%%%%%%%%%%%%%%%

% The best way to enter references is to use BibTeX:

\bibliographystyle{mnras}
\bibliography{references}

%%%%%%%%%%%%%%%%%%%%%%%%%%%%%%%%%%%%%%%%%%%%%%%%%%

%%%%%%%%%%%%%%%%% APPENDICES %%%%%%%%%%%%%%%%%%%%%

\appendix

% \section{section}
% \label{sec:sec}
\DIFdelbegin %DIFDELCMD < 

%DIFDELCMD < %%%
%DIF <  Teff plots
%DIFDELCMD < \begin{figure*}
%DIFDELCMD <   \centering
%DIFDELCMD <   \includegraphics[width=\textwidth]{plots/Teff.png}
%DIFDELCMD <   %%%
%DIFDELCMD < \caption{%
{%DIFAUXCMD
\DIFdelFL{$\Teff$ as a function of age for select $\gb$, with with }%DIFDELCMD < \nodm %%%
\DIFdelFL{models overplotted as thin lines. Stars undergo a sharp decrease in $\Teff$ as they leave the MS. The change in MS lifetimes due to ADM can be seen in the time difference between these features.
  }}
  %DIFAUXCMD
%DIFDELCMD < \label{fig:teff}
%DIFDELCMD < 

%DIFDELCMD < \end{figure*}
%DIFDELCMD < %%%
\DIFdelend 




%%%%%%%%%%%%%%%%%%%%%%%%%%%%%%%%%%%%%%%%%%%%%%%%%%



% Don't change these lines
\bsp	% typesetting comment

\label{lastpage}

\end{document}

% End of mnras_template.tex
