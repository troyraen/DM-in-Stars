% fs setup
\PassOptionsToPackage{pdfpagelabels=false}{hyperref}
% \documentclass[useAMS,fleqn,usenatbib]{mnras} % fleqn left-aligns equations
\documentclass[useAMS,usenatbib]{mnras}
\pdfoutput=1 %for arxiv
\setlength{\topmargin}{-0.3in}

\usepackage{graphicx}
\usepackage{amsmath,amssymb,amstext}
\usepackage[T1]{fontenc}
\usepackage{ae,aecompl}
\usepackage[utf8]{inputenc}
% \usepackage{newtxtext,newtxmath}
\usepackage[figure,figure*]{hypcap}
\usepackage[dvipsnames]{xcolor}
\usepackage{bm}

\usepackage{xparse}

%% MINE
%%%%%%%%%%%%%%%%%%%%%%%%%%%%%%%%%%%%%%%%%%%%%%%%%%%%%%%%%%%%%%%%%%%%%%
%
%               Macros for TeX/LaTeX documents
%
%
%%%%%%%%%%%%%%%%%%%%%%%%%%%%%%%%%%%%%%%%%%%%%%%%%%%%%%%%%%%%%%%%%%%%%%
\usepackage{xspace} % smart spaces after custom \newcommand

%-- Comments and questions
\newcommand{\arz}[1]{{\color{ForestGreen}\textbf{[Andrew: }\textbf{#1}]}}
\newcommand{\cb}[1]{{\color{TealBlue}\textbf{[Carlos: }\textbf{#1}]}}
\newcommand{\hrm}[1]{{\color{Mulberry}\textbf{[Héctor: }\textbf{#1}]}}
\newcommand{\tjh}[1]{{\color{Red}\textbf{[Travis: }\textbf{#1}]}}
% \newcommand{\fvdb}[1]{{\color{BurntOrange}\textbf{[Frank: }\textbf{#1}]}}
% \newcommand{\cs}[1]{{\color{Lavender}\textbf{[ChadS: }\textbf{#1}]}}
\newcommand{\kw}[1]{{\color{Mulberry}\textbf{[Kuan: }\textbf{#1}]}}
\newcommand{\tjr}[1]{{\color{Brown}\textbf{[Troy: }\textbf{#1}]}}

\newcommand{\cbq}[0]{{\color{TealBlue}\textbf{[Question for Carlos: ]}}}
\newcommand{\todo}[1]{{\color{Mulberry}\textbf{[TO DO: }\textbf{#1}]}}
\newcommand{\qus}[1]{{\color{BrickRed}\textbf{[Q: }\textbf{#1}]}}


%-- Units
\newcommand{\cmsq}{\ \mathrm{cm}^2}
\newcommand{\cmcinv}{\ \mathrm{cm}^{-3}}
\newcommand{\gev}{\ \mathrm{GeV}}
\newcommand{\K}{\ \mathrm{K}}
\newcommand{\kms}{\ \mathrm{km\ s}^{-1}}
\newcommand{\Msun}{\ \mathrm{M}_{\odot}} % Msun
\newcommand{\sinv}{\ \mathrm{s}^{-1}}


%-- Variables
% must be used within math mode unless command includes $$
% \newcommand{\mp}{$m_p$}
\newcommand{\Csun}{C_{\odot}}
\newcommand{\Cx}{C_{\mathrm{DM}}}
\newcommand{\epsCNO}{\epsilon_{\mathrm{CNO}}}
\newcommand{\epsnuc}{\epsilon_{\mathrm{nuc}}}
\newcommand{\epspp}{\epsilon_{\mathrm{pp}}}
\newcommand{\epsx}{\epsilon_{\mathrm{DM}}}
\newcommand{\gb}{\Gamma_\mathrm{B}}
\newcommand{\gbzero}{\Gamma_\mathrm{B} = 0}
\newcommand{\gbone}{\Gamma_\mathrm{B} = 1}
\newcommand{\gbpow}[1]{\Gamma_\mathrm{B} = 10^{#1}}
\newcommand{\LvsT}{log($L$)-log($\Teff$)\xspace}
\newcommand{\lx}{l_{\mathrm{DM}}}
\newcommand{\mprot}{m_{\mathrm{p}}}
\newcommand{\mrange}{$0.9 \leq M_{\star}/\mathrm{M}_{\odot} \leq 5.0$\xspace}
\newcommand{\mrangelow}{$0.9 \leq M_{\star}/\mathrm{M}_{\odot} \lesssim 1.3$\xspace}
\newcommand{\mrangehigh}{$1.3 \lesssim M_{\star}/\mathrm{M}_{\odot} \leq 5.0$\xspace}
\newcommand{\Mstar}{M_{\star}}
\newcommand{\mx}{m_{\mathrm{DM}}}
\newcommand{\nprot}{n_{\mathrm{p}}}
\newcommand{\Nx}{N_{\mathrm{DM}}}
\newcommand{\nx}{n_{\mathrm{DM}}}
\newcommand{\rhox}{\rho_{\mathrm{DM}}}
\newcommand{\Rstar}{R_{\star}}
\newcommand{\rx}{r_{\mathrm{DM}}}
\newcommand{\sigxn}{\sigma_{\mathrm{n}}}
\newcommand{\sigxp}{\sigma_{\mathrm{p}}}
\newcommand{\Tc}{T_\mathrm{c}}
\newcommand{\Teff}{T_{\mathrm{eff}}}
\newcommand{\Tx}{T_{\mathrm{DM}}}

%-- Other
\newcommand{\mesa}{\texttt{MESA}\xspace}
\newcommand{\mist}{\texttt{MIST}\xspace}
\newcommand{\nodm}{\texttt{NoDM}\xspace}


\usepackage{url}

%% ENDMINE

%%%%%%%%%%%%%%%%%%% TITLE PAGE %%%%%%%%%%%%%%%%%%%

% Title of the paper, and the short title which is used in the headers.
% Keep the title short and informative.
\title[Asymmetric Dark Matter in Stars]{The Effects of Asymmetric Dark Matter on Stellar Evolution I: Spin-Dependent Scattering}

% The list of authors, and the short list which is used in the headers.
% If you need two or more lines of authors, add an extra line using \newauthor
\author[T.J. Raen et al.]{%
Troy J. Raen,$^{1}$\thanks{E-mail: troy.raen@pitt.edu},
Héctor Martínez-Rodríguez$^{1}$,
Andrew R. Zentner$^{1}$,
Travis J. Hurst$^{2}$, \newauthor
Carles Badenes$^{1}$,
and Rachel Tao$^{3}$
\vspace*{12pt}
\\
% List of institutions
$^{1}$Department of Physics and Astronomy \& Pittsburgh Particle Physics, Astrophysics, and Cosmology Center (Pitt PACC), University of Pittsburgh, Pittsburgh, PA 15260, USA\\
$^{2}$Department of Mathematics and Physics, Colorado State University -  Pueblo, Pueblo, C0, 81001 \\%Fixed my contact info. TJH
$^{3}$Department of Physics, Emory University, Atlanta, GA 30322
}
% These dates will be filled out by the publisher
\date{\today}

% Enter the current year, for the copyright statements etc.
\pubyear{2019}

% Don't change these lines
\begin{document}
\label{firstpage}
\pagerange{\pageref{firstpage}--\pageref{lastpage}}
\maketitle

% fe setup

% Abstract of the paper
\begin{abstract}
We study the effects of Dark Matter on the evolution of stars with masses 0.8-5.0$\Msun$ and find that, in certain environments, main sequence lifetimes can be dramatically altered. \tjr{Finish writing abstract...}
\end{abstract}

% Select between one and six entries from the list of approved keywords.
% Don't make up new ones.
\begin{keywords}
keyword1 -- keyword2 -- keyword3
\end{keywords}

%%%%%%%%%%%%%%%%%%%%%%%%%%%%%%%%%%%%%%%%%%%%%%%%%%







%%%%%%%%%%%%%%%%% BODY OF PAPER %%%%%%%%%%%%%%%%%%

\arz{Units should be set in roman type, such as $12~\mathrm{GeV/cm}^3$ and so on. Or,
you can do what I do, which is to use macros, such as $15~\gev$ and $200~\kms$.}

\todo{find other stellar textbook(s) to cite.}


\vspace{10mm}

\tjr{GENERAL NOTES: The plots are ready for review. I need to update the text before it is worth reviewing.}

\section{Introduction}
\label{sec:intro}

  In the standard cosmological picture, galaxies live preferentially within large clusterings of dark matter (DM) particles called halos. If DM can scatter with standard model particles, it can be captured by stars moving through these halos. Once captured, continued scattering contributes to energy transport within the star, potentially altering its evolution \citep{Spergel1985EffectInterior, Zentner2011AsymmetricDwarfs} \tjr{get a few more references}. The significance of this energy transport depends on the following (in addition to the stellar properties): (1) DM mass, $\mx$; (2) DM-nucleon scattering cross section, $\sigxn$; and (3) total number of DM particles captured by a star, $\Nx$, which itself depends on $\sigxn$ and properties of the local DM environment which we parameterize in \ref{sec:props}. We study the effects of energy transport by asymmetric dark matter (ADM, see below) \qus{quote mass and xsection values here?} in stars of mass \mrange using the publicly available code Modules for Experiments in Stellar Astrophysics (\mesa) \citep{Paxton2011ModulesMESA}.

  Evidence that $\sim$84\% of the matter in the universe is in an unknown form called Dark Matter (DM) \qus{capitalize dark matter? re-reference abbreviation?} is abundant and varied, ranging from the anisotropy of the microwave background radiation to the structures of galaxies. However, it has never been directly observed and much remains to be discovered about its fundamental nature. For several decades, the leading candidate has been weakly-interacting massive particles (WIMPs). Classic WIMPs are thermal relics which would limit the number of particles in a star as DM capture came to equilibrium with WIMP mutual annihilation. They are also predicted to have a mass 1-2 orders of magnitude above the proton mass which would limit the effectiveness of their energy transport by restricting their orbital range to a relatively homogeneous region in the core. However, despite ongoing terrestrial direct detection experiments, DM has not been observed and the available parameter space for classic WIMPs is shrinking rapidly \citep{Amole16}\tjr{Still need to refer to a constraint paper.}, prompting research into other models.

  ADM predicts a DM particle that is less massive than typical WIMPs and is a product of primordial asymmetry, not thermal processes, and so does not self-annihilate \qus{correct terminology?} - a scenario that allows a star to build up a large quantity of DM with orbital paths crossing distinct regions of the star. Thus, the energy transported by ADM can more drastically alter the stellar evolution. In this paper we further narrow our study to spin-dependent ADM-nucleus scattering because it leads to behaviors that are qualitatively distinct from spin-independent scattering. We will present results for spin-independent scattering in a forthcoming manuscript.

  % \arz{How about this edit (starting with a new paragraph)?}
  %   The degree of stellar cooling induced by dark matter
  %   depends upon the following DM properties:
  %   1) the WIMP mass $m_x$;
  %   2) the WIMP-nucleon scattering cross section $\sigma_{xn}$;
  %   and 3) the total number of WIMPs captured by a star $N_x$.
  %   We will characterize stellar cooling using these parameters.
  %   As we will show below, our results are most relevant to a class
  %   of dark matter models known as {\em asymmetric dark matter} (ADM)
  %   \arz{Need some generic ADM citations here, start by looking in my paper.}
  %   The number of standard WIMP dark matter particles that can be captured
  %   within a star is limited by the fact that WIMP capture will come to
  %   equilibrium with WIMP mutual annihilation within the star. In ADM models,
  %   the dark matter does not annihilate because the dark matter consists entirely
  %   of dark matter particles and is completely devoid of dark matter antiparticles.
  %   For this reason, $N_x$ can grow very large in ADM models, enabling the
  %   effects of dark matter on stellar evolution to grow correspondingly large
  %   over the lifetime of a star. Consequently, our proposal is to use
  %   astrophysical studies of stellar populations to constrain ADM. In this first
  %   paper, we further narrow our study to spin-dependent ADM-nucleus scattering
  %   because spin-dependent and spin-independent scattering lead to qualitatively
  %   distinct behaviors. We will present results for spin-independent scattering
  %   in a forthcoming manuscript.
  %   \arz{My proposal is to cut the following if you like the above.}
  %   In so doing, we remain agnostic as to the fundamental theory of the
  %   dark matter and quantify
  %   Therefore we remain agnostic
  %   \tjr{this sounds good, but I think is not technically what I mean. Is this a common usage?}
  %   to most of the details of the WIMP model and focus on the relevant parameter space that
  %   is both allowed by experiment and effective at transporting energy in the star.

  While we cannot directly observe stellar interiors, standard models have been developed and refined over the last century that match observations quite well. They are based on solving a set of coupled differential equations that describe energy and mass conservation, energy transport, and hydrostatic equilibrium using equations of state, tabulated opacities, and the (evolving) compositions of stellar interiors (see \citealt{Pols1990StellarEvolution} for details).
  \arz{I would also cite some of the classic textbooks here, because most of what you are discussing is available in standard textbooks on stellar evolution. My favorite book is the bible of stellar evolution, Kippenhahn and Weigert, but there are several good books on this topic.}
  % For future reference, we list several approximate scaling relations which are followed by MS stars in the standards models:
  % \begin{align}
  %   L \propto M^(3.5 to 4) \\
  %   R \propto M \\
  %   tau \propto M^-(2.5 to 3) \\
  %   Tc \propto M^0.57 pp, M^0.21 cno (pols page 132)
  % \end{align}

  Stars spend most of their lifetimes on the main sequence (MS), a near equilibrium state powered by hydrogen burning in the core. This conversion of hydrogen to helium happens via two main channels: the proton–proton (pp-) chain and the carbon–nitrogen–oxygen (CNO) cycle. The CNO cycle is much more sensitive to the temperature, as can be seen by the approximate scaling relations of the burning rates:
  \begin{align}
    \epspp &\propto T^4 \\
    \epsCNO &\propto T^{18}
  \end{align}
  This has several important consequences:
  \begin{enumerate}
    \item The pp-chain (CNO cycle) dominates in stars with central temperatures, $\Tc$, less (greater) than $\approx 2 \times 10^7 \K$.

    \item The stable stellar structures in the two regimes are different. In CNO-dominated stars, radiative transport is insufficient to carry such large amounts of energy away from the central burning region, requiring these stars to have convective cores. In contrast, pp-dominated stars are stable with purely radiative cores.

    \item Smaller energy production rates in pp-dominated stars also means that DM energy transport can be significant at much lower values than in CNO-dominated stars.
  \end{enumerate}

  $\Tc$ generally increases with stellar mass, and the temperature threshold in (i) is equivalent to $\Mstar \approx 1.3 \Msun$. This means that our mass range includes two qualitatively distinct groups, which we will refer to as low-mass stars (\mrangelow) and high-mass stars (\mrangehigh) for the purposes of this paper \qus{Is this ok or should I really call them low-mass and intermediate-mass stars?}.

  We present our final results in the form of Hertzsprung–Russell (HR) diagrams\footnote{Note that an observer's HR diagram plots color versus magnitude while theorists use $\Teff$ and $L$ which captures the same information. \qus{do I need this footnote?}}. In Figure \ref{fig:tracks} we plot stellar tracks (properties are functions of time for a given stellar mass). In Figure \ref{fig:isos_cb} we plot isochrones (properties are functions of stellar mass for a given age).

  The observed luminosity, $L$, and effective temperature, $\Teff$, of a MS star remain roughly constant over human time scales; however, these observables do vary as a function of mass, making star clusters feasible testing grounds for DM constraints. All stars in a cluster live in the same DM environment and are assumed to have formed from a single molecular cloud, meaning they have approximately the same age and metallicity. Given a stellar evolution model and an initial mass function for the cluster, its properties can be predicted and isochrones compared to observations. We leave quantitative constraints to future work since they will require in-depth analysis of stellar demographics \qus{and environments?}.

% \subsection{Observing Stellar Evolution}
%
%   \tjr{Are there other parts of the paper for which this is also true?}
%   \arz{The information in this section is all correct and you should hang on to it
%   for the purposes of writing your thesis document, but... I would say that this
%   information is not necessary in a journal article. So, I would get rid of the
%   subsection heading and reduce this section to one, short paragraph.
%   I think that we can say that
%   we cast our final results into the form of H-R or color-magnitude diagrams and that
%   we defer taking this all the way to a constraint for a future paper. The reason for
%   deferring is simply that extracting an actual constraint from observational data
%   requires a great deal of work on stellar demographics that suffices to constitute a
%   publication of its own.}
%
%   Ultimately, the effects we calculate can be compared to observations to further constrain DM properties. We leave this comparison to future work, but will briefly describe the process in order to focus our results.
%
%   Stellar evolution is best observed using Hertzsprung–Russell (HR) diagrams\footnote{Note that an observer's HR diagram plots color versus magnitude while theorists use T$_{\rm{eff}}$ and L which captures the same information.} of star clusters. All stars in the cluster are assumed to have formed from a single molecular cloud and therefore to have approximately the same metallicity and age. For this reason these diagrams are also called isochrones. Given a stellar evolution model and an initial mass function for the cluster, their properties can be predicted and compared to observations.
%
%   Since the MS is a near equilibrium state, with gravitational contraction balanced by outward radiation pressure from nuclear burning, the observed luminosity, L, and effective temperature, T$_{\rm{eff}}$, remain roughly constant at values which increase with the star's total mass. Then MS stars lie roughly along a line with negative slope in an HR diagram (see Figure \ref{fig:stellarHR}).
%
%   As the core hydrogen supply depletes, the local burning rate must also decrease and the star leaves the MS. With less radiation pressure to counter gravity, the core contracts and the temperature rises. Once the temperature just outside the depleted core is sufficiently high, the hydrogen in that region ignites and the star enters a period of hydrogen shell burning. (This process happens quickly in  high mass stars and gradually in low mass stars, see Section \ref{sec:results} for details.) The shell burning region acts as a mirror and evolve up and to the right as they leave the MS \tjr{needs better explanation}. High mass stars have much higher central burning rates and so they burn through their hydrogen supply much more quickly. MS lifetimes are then inversely proportional to mass, and so a cluster's age can be determined by the location of this MS turn-off on the isochrone.
%
%   % %--------------------
%   % % stellar HR plots
%   % \begin{figure*}
%   % \centering
%   %   \includegraphics[width=\textwidth]
%   % %   {plots/HR_1_and_3p5_Msun.png}
%   % {plots/HR.png}
%   %   \caption{\label{fig:stellarHR}HR diagrams of the evolution of $1\Msun$ and $3.5\Msun$ stars. Only NoDM and c6 are shown for brevity. Marked positions correspond to the time the central hydrogen mass fraction falls below a threshold: ZAMS: 0.71, IAMS: 0.3, H-3: $10^-3$. (ZAMS and IAMS are used in Dotter.) Both models settle onto the MS at the same position in the HR plane, but move through the MS in different ways. NOTE: The data is cut: removed most of pre-MS and after MS turnoff. These cuts make it easier to see what I'm trying to show, but I'm not sure that my specific choices were the best. We should discuss.
%   %   }
%   % \end{figure*}
%   % %--------------------


%--------------
\section{Dark Matter Properties and Capture in Stars}
\label{sec:props}

  \qus{In much of this section I'm using 'DM' instead of 'ADM' because I think this stuff applies to dark matter in general and not just ADM. Is that right? And/or should I just stick with ADM throughout for consistency?}

  Probing the parameter space of ADM with simulations of stellar evolution is computationally expensive. Consequently, we show results for a representative set of ADM parameters chosen to: (1) make the effects of ADM on stellar evolution significant; and (2) remain consistent with contemporary constraints on dark matter properties. For our models we choose $\mx = 5 \gev\ (\approx 5m_p)$ and a spin-dependent scattering cross section\footnote{We take $\sigxn \rightarrow \sigxp$ since the overwhelming majority of spin-dependent DM scattering events in MS stars are with protons.} of $\sigxp = 10^{-37} \cmsq$. We assume that ADM self-interactions are negligible throughout; however, it is likely that self-interactions will lead to enhanced cooling \citep[e.g.,][]{Zentner2009High-energySun} and exploring such models would constitute a potentially interesting follow-up to this work.

  \tjr{Need something here on current m$_x$ and $\sigma_x$ constraints.}
  \arz{I'm guessing you got this from Brian already, but the Cosmic Visions report
  from last year gives a nice summary along with links to all of the relevant
  experiments.}

  The energy transported by ADM of this mass and cross section in a given star then depends on the amount of ADM captured, given by $\Nx = \Cx t$ where $\Cx$ is the capture rate and $t$ is time. We use the capture rate from \citet{Zentner2011AsymmetricDwarfs} \citep[see also][]{Gould1992CosmologicalAnnihilations, Zentner2009High-energySun}:

  \begin{align}
  \begin{split}
    \label{eq:capturerate}
    \Cx =\ & \Csun
    \Big(\frac{\rhox}{0.4 \gev \cmcinv}\Big)
    \Big(\frac{270 \kms}{\bar{v}}\Big) \\
    & \times \Big(\frac{\sigxp}{10^{-43} \cmsq}\Big)
    \Big(\frac{v_{esc}}{618 \kms}\Big)^2
    \Big(\frac{\Mstar}{\Msun}\Big)
  \end{split}
  \end{align}
  where $\Csun = 5 \times 10^{21} (5 \gev/ \mx) \sinv$ is the spin dependent capture
  rate in the sun, $\rhox$ is the DM density in the star's environment,
  $\bar{v}$ is the typical speed at infinity of the infalling DM,
  and v$_{esc}$ is the escape speed from the surface of the star.
  % This capture rate depends on both the properties of the ADM and the dark matter halo in which the star lives.

  Both $\rhox$ and $\bar{v}$ are properties of the local stellar environment (not the dark matter itself) and are degenerate with each other in Eq.~(\ref{eq:capturerate}) \qus{isn't it more general than just this equation? I thought these two things couldn't be measured independently. Or maybe that's only true for this particular method?}. Therefore it is convenient to specify \tjr{parameterize?} the star's DM environment by an overall boost factor \citep{Zentner2011AsymmetricDwarfs,Hurst2015},

  \begin{equation}
  \gb = \Big(\frac{\rhox}{0.4 \gev \cmcinv}\Big) \Big(\frac{270 \kms}{\bar{v}}\Big).
  \label{eq:gammab}
  \end{equation}
  $\gb$ then specifies $\Cx$ relative to the rate that would be realized in the solar neighborhood for the same star.
  % It is most natural to think of $\Gamma_B$ as quantifying the dark matter environment. For example,
  $\gbzero$ describes a stellar environment with no DM (hereafter referred to as `standard models' and labeled `NoDM'), and $\gbone$ describes the solar neighborhood. A value of $\gbpow{2}$ may specify an environment in which the dark matter density is 100 times that in the solar neighborhood at the same velocity dispersion, an environment in which the velocity dispersion is 1/100 that of the solar neighborhood at the same density, or any of an infinite number of other possible combinations.

  Local Group dwarf galaxies are constrained to have both densities that are considerably higher than, and velocity dispersions that are considerably lower than, that found near the sun. Consequently, $\gb$ can achieve extremely large values in dwarf galaxies, approaching $\gbpow{6}$
  \arz{To do item for Zentner to cite the proper literature for dwarf galaxy structures.}.
  \arz{To do item for Zentner to estimate GammaB in high-redshift galaxies.}
  Finally, we note that values of $\gb \ne 1$ can also be realized through dark matter physics, particularly dark matter self-interactions \citep{Zentner2009High-energySun}.
  % \arz{It is very uncertain whether or not globulars have any DM, so let's not discuss
  % them here and just get rid of everything below here.}
  % Typical globular clusters (low v) and dwarf galaxies (high rho)
  % \tjr{needs better explanation} can have high DM concentrations and/or low infall velocities so that $\Gamma_B \approx 10^3 - 10^4$ (Hurst and Zentner, 2017 \tjr{Can't find this paper on the arxiv.} \arz{There is no such paper! I don't think a citation is needed here.}. The most extreme environments we can expect to find will have $\Gamma_B \approx 10^6 $ \tjr{reference?}.


%----------------------------------------------
\section{Simulating Stellar Evolution}
\label{sec:methods}

  We study the evolution of \mrange stars through the thermal pulse (or equivalent) phase using the publicly available code Modules for Experiments in Stellar Astrophysics (\mesa), \citep{Paxton2011ModulesMESA}. \mesa models stellar evolution by simultaneously solving the coupled stellar structure and composition equations. We base our models on the \mesa inlist used in \citet{Choi2016mesaModels}\footnote{\url{http://waps.cfa.harvard.edu/MIST/}} (see also \citealt{Dotter2016MesaIsochrones}); however, we turn off rotation and diffusion and use a hard net \tjr{explain what this means}. \arz{You should probably comment that this choice makes our models and our isochrones comparable to those published by Dotter. This is a nice feature of making this choice and a good motivation for this choice of inlist.} Our results are robust to these changes. \tjr{double check diffusion and hard net results}. This choice makes our models and our isochrones comparable to those published by Dotter. We use a solar metallicity of z=0.014. DM energy transport is calculated as described in \ref{sub:energytransport} and passed to \mesa using the native \texttt{other\_energy\_implicit} hook which includes this energy self-consistently when solving the coupled equations.

  The size of each time step in a \mesa model can vary by orders of magnitude, and so output for different \mesa runs may contain information at drastically different times. To generate isochrones we must interpolate output from a range of initial stellar masses, with all other parameters held constant. We accomplish this using code written by \citet{Dotter2016MesaIsochrones}\footnote{\url{https://github.com/aarondotter/iso}}. It takes a set of \mesa runs and uses key evolutionary phases to guide the interpolation. This helps to ensure that phases with shorter time scales are properly represented in the interpolation.

%---------
\subsection{Energy Transport by Dark Matter}
\label{sub:energytransport}

  The energy transported by captured ADM can, in principle, be computed by solving the Boltzmann equation; however, this strategy is too computationally intensive to combine with a full-scale simulation of the evolution of stellar structure. To reduce the computational costs of our simulations, we estimate ADM energy transport using the approximations of \citet{Spergel1985EffectInterior}. In particular, we assume a Maxwellian phase-space distribution for the ADM and calculate an orbit-averaged temperature, $\Tx$, by requiring that the distribution satisfy the first moment of the Boltzmann equation. This amounts to a requirement on energy conservation: ADM should neither inject nor remove a net energy from the star. The rate of energy transfer (per unit mass) from dark matter to protons is then
  % \tjr{use \ is full space, '\,' '\.' . can also use begin{eqnarray} }
  %
  \begin{align}
  \begin{split}
  \epsx(r) =\ & 8\ \sqrt[]{\frac{2}{\pi}} \frac{\nx(r) n_p(r)}{\rho(r)} \frac{\mx m_p}{(\mx+m_p)^2} \sigxp\\
  & \times \Big(\frac{m_p k \Tx + \mx k T(r)}{\mx m_p}\Big)^{1/2} k (\Tx - T(r)),
  \label{eq:xheat}
  \end{split}
  \end{align}
  %
  where $n(r)$ is a number density, $\rho(r)$ is the mass density, $k$ is Boltzmann's constant, and the subscript $p$ refers to protons. \citep[See][for a detailed calculation]{Spergel1985EffectInterior}.

  Generally, $n_p$, $\nx$, and $T$ all peak at the center (exceptions noted below), so the energy transport is most efficient here. $\nx$ increases with $\Nx$, so we can expect the effects to increase with both $\gb$ and stellar age through the MS, while hydrogen is abundant. As a star leaves the MS, $n_p$ drops in the core and spin-dependent ADM energy transport is greatly diminished because there are relatively few protons left to scatter with. A standard MS temperature profile decreases monotonically with distance from the star's center, but it can become inverted when ADM moves large amounts of energy away from the center (requires $\gb \gg 1$) \qus{should I mention any standard conditions that cause this? degeneracy?}.

  The sign of $\epsx(r)$ is given by the final term in (\ref{eq:xheat}), $\Tx - T(r)$, which is used to define an ADM characteristic radius, $\rx$, implicitly as
  %
  \begin{align}
    T(\rx) = \Tx.
  \end{align}
  %
  Then dark matter takes energy from $r < \rx$ and deposits it at $r > \rx$ for a standard MS temperature profile. With our chosen ADM parameters we see typical values:
  %
  \begin{align}
    \rx & \sim \mathcal{O}(0.1 \Rstar) \\
    \lx = (\sigxp n_p)^{-1} & \sim\mathcal{O}(1 \Rstar)
  \end{align}
  \qus{is this a right/good way to write this? also, should $\Rstar$ be in Roman font here?}
  \todo{check that the data agree with these estimates}
  %
  where $\lx$ is the ADM mean free path (implying that it completes several orbits between scattering events). These values allow dark matter to travel much larger distances than photons or baryons \qus{right word?} (which have $l \lesssim 10^{-10} \Rstar$) and to traverse qualitatively distinct regions of the star. \qus{should I mention anything else here? like seeing a significant temperature gradient or something related to the convection effects?}

  % \arz{The preceding statement is probably not true, right? Near $r=0$, the temperature gradient is very small, so energy transport probably is not maximally efficient at the center of the star.} \tjr{What I mean is that the magnitude of the extra heat is largest at r=0, see Fig \ref{fig:m1p0_a}, top right plot. This is generally, but not always, true. Is it the word "efficient" that's a problem?}
  \todo{orphaned: dm probes temp diffs over large portion of star. so temp gradient shallow at center dm energy transport can still be efficient. contrary to standard stellar evolution.}

  \todo{check that old plots are xluminosity and not xheat.}


%------------------------------------------
\section{Results}
\label{sec:results}

  % mstau plot
  \begin{figure*}
    \centering
    \includegraphics[width=\textwidth]{plots/mstau.png}
    \caption{CHANGES: yaxis label remove 'MS', don't italicize 'NoDM'. xaxis label -> Star Mass, values -> 1,2,.. need more tick marks. Colorbar rotate gammaB to face up.
    Change in MS lifetime relative to NoDM model. The presence of DM generally shortens the MS lifetime.
    Differences in the two categories of stars in this mass range can be seen here in the non-monotonicity around roughly 1.3 Msun.
    }
    \label{fig:mstau}
  \end{figure*}

  % Teff plot
  \begin{figure*}
    \centering
    \includegraphics[width=\textwidth]{plots/Teff.png}
    \caption{CHANGES: increase all font sizes. yaxis label Teff italicized / K. xaxis lower limit 7. Perhaps remove c4 panel and make the 3rd panel a zoom-in that includes c4?}
    \label{fig:Teff}
  \end{figure*}

  % stellar tracks plots
  \begin{figure*}
    \centering
    \includegraphics[width=\textwidth]{plots/tracks.png}
    \caption{CHANGES: all font sizes bigger except plot labels. axis labels log(variable (italic)/ unit). Add and label ZAMS line.}
    \label{fig:tracks}
  \end{figure*}

  % isochrones plots
  \begin{figure*}
    \centering
    \includegraphics[width=\textwidth]{plots/isos_cb.png}
    \caption{CHANGES: all font sizes bigger except plot labels. axis and colorbar labels log(variable (italic)/ unit). Consider zoom in of lower right on c6 plot. Consider labelling lines with ages. Need to make it more obvious that last 2 isochrones are missing from right panel.
    CAPTION: Triangles are 3.5$\Msun$, circles are 1.0$\Msun$. There is missing data where the isochrone code did not interpolate. This is due to non-monotonicity in the initial mass - age relation of a given EEP (equivalent evolutionary phase), which is a known problem that Dotter discusses in his paper.}
    \label{fig:isos_cb}
  \end{figure*}


  We find that, for stars significantly affected by ADM energy transport, the general effect is to shorten MS lifetimes (see Figure \ref{fig:mstau}) so that, for a stellar populations of a given age, the MS turn-off happens at a lower mass and the stars evolve through the sub-giant branch at a lower luminosity (see Figure \ref{fig:isos}) when ADM cooling is operative. This causes isochrones of clusters in high $\gb$ environments to appear older than their standard model counterparts. This becomes noticeable in the highest $\gb$ environments around 0.2 Gyr as stars with $\rm{M} \approx 4 \Msun$ begin to leave the MS. Prior to this the stars have not had enough time to capture a sufficient number of WIMPs for ADM-driven energy transport to be a significant contribution to the overall energy transfer within the star.

\subsection{High-Mass Stars: \mrangehigh}
\label{sub:highmass}

  % 3.5Msun, energy and convection
  \begin{figure*}
    \centering
    \includegraphics[width=\textwidth]{plots/m3p5.png}
    \caption{
    CHANGES: main title ''. column titles 'energy production and transport', 'convection'. yaxes label with units, remove them from legend. put 0 line on epschi for c0. fix the scale? xaxis label m(<r)/Msun. reduce linewidth. increase space between rows, maybe decrease height of plots.
    CAPTION: 3.5Msun. Cboost colors are the same as the previous plots. Time moves down the page. Left column (note large scale changes): eps nuc is on the left axis, DM energy transport is on the right. Negative (positive) values of epschi indicate DM is removing (depositing) energy. Right column: convective mixing (log(D) where D is diffusion coefficients of convection + overshoot.). As the radiative core grows in the c6 model, the burning gradually shifts outward to shell burning, following the inner edge of the convective shell. The hydrogen supply is not replenished outside the convective zone. NOTES: I'll put the grey arrows in exactly the right place once we settle on which time steps to show. The $h1_c$ legend is just a check to make sure the specific profile used is close enough to the one I wanted. I will remove this for the final plot.}
    \label{fig:m3p5}
  \end{figure*}

  In standard models, MS stars with $\Mstar \gtrsim 1.3 \Msun$ are powered primarily by the CNO cycle which has two important consequences: (1) the burning rate is much higher than in pp-dominated stars and is extremely sensitive to the temperature, and (2) the core must be convective in order to carry the large energy flux. The convection extends beyond the burning region, giving the star access to a larger quantity of fuel by mixing fresh hydrogen into the center. As a result, convection extends the MS lifetime\footnote{This extension is subdominant. Central temperatures increase with stellar mass, causing the burning rates to increase dramatically which shortens standard model stellar lifetimes so they decrease approximately as $\tau \propto \Mstar^{-2.5}$.}. Once hydrogen throughout the convective zone is depleted, the burning rate rapidly decreases and the star loses more energy at its surface than is being generated by burning. Gravity temporarily dominates and the entire star contracts until the temperature increases enough to ignite hydrogen in a shell just outside the depleted core. This can be seen in the `NoDM' models (light blue \qus{green? yellow?}) in Figure \ref{fig:m3p5}. \qus{why does the convective region shrink while burning rate increases?} The contraction causes a temporary increase in
  $\Teff$ and a feature in the star's HR diagram called the convective hook (see Figures \ref{fig:tracks} and \ref{fig:isos_cb}). See \citet{Pols1990StellarEvolution} for more details.

  If a star captures enough ADM, the combination of dark matter + radiative energy transport becomes sufficient to carry the flux from nuclear burning. Convection disappears from the center first (where ADM energy transport is most efficient) and retreats away from the core, into a narrowing shell. Without convective mixing, the central hydrogen supply depletes and so the burning also shifts into a shell, following the lower boundary of the convective zone (Figure \ref{fig:m3p5}). The result is that these stars leave the MS earlier and at a lower luminosity, and they skip the convective hook altogether (Figure \ref{fig:tracks}). This makes the isochrones appear older than their `NoDM' counterparts (Figure \ref{fig:isos_cb}). The effects disappear as $\Mstar$ approaches $5 \Msun$ because stellar lifetimes become too short for a sufficient amount of ADM to build up.


\subsection{Low-Mass Stars: \mrangelow}
\label{sub:lowmass}

  % 1.0Msun, energy and temperature
  \begin{figure*}
    \centering
    \includegraphics[width=\textwidth]{plots/m1p0c3.png}
    \caption{1.0Msun. Cboost colors are the same as the previous plots. Time moves down the page. Left column: (same as previous plot) eps nuc is on the left axis, DM energy transport is on the right (note large scale changes here). Right column: temperature (DM temperature marked as thin dotted line.
    DM energy transport decreases the burning in the core and pushes the burning into a shell more quickly than the reference models.
    }
    \label{fig:m1p0_a}
  \end{figure*}

  % 1.0Msun, energy, temperature, density profiles
  \begin{figure*}
    \centering
    \includegraphics[width=\textwidth]{plots/m1p0c6.png}
    \caption{Star mass 1.0Msun, $\Gamma_B = 10^6$ (colors are different than previous plots). Time moves down the page with rows correspond to times from figure \ref{fig:m1p0_kipp}. 'Time1-5' were chosen to show the progression and extremes of DM energy transport during one oscillation cycle. Scales are fixed for ease of comparison. 'Degen' shows conditions after the oscillations stop, when electron degeneracy pressure is supporting the core (note scale changes). Left column has $\epsilon_{nuc}$ on left axis (red) and DM energy transport on the right (blue). Right column has logT on left axis (red, with T$_x$ marked with thin dotted line) and logRho on the right axis (blue).
    %   Time 1: 3.1736e8 yrs, xheat$_c$ is maximum, T$_c$ and $\epsilon_c$ are increasing. Time 2: 3.229e8 yrs, xheat$_c$ is minimum, T$_c$ and $\epsilon_c$ are decreasing (this should be first time shown in plot). Time 3: 3.288e8 yrs, T$_c$ is minimum, xheat$_c$ and $\epsilon_c$ are increasing. Time 4: 3.81e8 yrs, core degeneracy has become significant.
    }
    \label{fig:m1p0_profs}
  \end{figure*}

  % 1.0Msun c6 Kippenhahn with profile times labeled
  \begin{figure*}
    \centering
    \includegraphics[width=\textwidth]{plots/m1p0c6_age.png}
    \caption{Star mass 1.0Msun, $\Gamma_B = 10^6$.  Figure \ref{fig:m1p0_profs} shows profiles of the star at the times marked here by vertical black lines.
    % Star mass 1.0Msun, c6, xaxis is age. Top plot shows core burning volume (red shaded area, left axis. Add color variations to show intensity) and L (right axis. Important lines are in the middle, red (log L) and white (log LH). Top and bottom lines should be removed from plot.
    Yellow and red show burning intensity as a function of age and mass coordinate (left axis) (I believe the burning thresholds are 1 erg/g/s and 10 erg/g/s respectively, but I'm a little confused about the mesa output and I need to check with Héctor or Carles.) Darker yellow and red lines mark the burning extent of the NoDM model over this time period.
    Blue line is luminosity (right axis). NOTE: Dotted blue line is hydrogen burning luminosity (I will probably remove this line for the final plot, but I want to check with Carles to make sure the behavior makes sense. I think the explanation is: When LH>L star is producing more energy than it is losing, causing it to expand, and . I have a separate plot of the radius that we can look at if needed. Actually could add logR to this plot.)
    %   (bottom plot). Bottom plot shows R$_{\star}$ (yellow, left axis) and r(m<0.01). As the core contracts, the envelope expands.
    }
    \label{fig:m1p0_kipp}
  \end{figure*}

  Standard model stars in this mass range have relatively low central temperatures and so are powered primarily by the pp-chain, which is much less sensitive to the temperature. This means the burning does not peak as strongly at the center and radiative transport is sufficient to carry the energy flux, so the core is not convective. Without convective mixing, hydrogen depletes first at the very center and the burning shifts outward gradually, avoiding the instability that causes the convective hook in standard CNO-powered stars. This behavior is very similar to high-mass stars that collect large amounts of ADM, which contributes to isochrones appearing older as $\gb$ gets large.

  Since the burning rate is much lower, the same number of captured dark matter particles have a larger effect in this mass range. As stars with $\gb$ as low as $10^2$ \todo{check this number for several masses} enter the MS, ADM is already transporting a significant amount of energy away from the center ($\epsx \approx \epspp$ near $r=0$) and depositing it in a shell at $m(<r) \approx 0.1 \Msun$. This causes the temperature, and therefore the burning rate, to be lower at the center and higher in a shell relative to standard models. Stars in environments with $\gb \lesssim 10^4$ are able to remain stable in this configuration. See Figure \ref{fig:m1p0_a}.
  % and their evolution is not significantly altered \tjr{looks significant in MStau plot...?}.

  Models with $\gb \gtrsim 10^4$ capture enough ADM so that $\epsx \gg \epspp$ near $r=0$ which destabilizes the core and sets up a series of oscillations. \todo{check different masses, particularly 0.8Msun for which the effect may extend down to c4.}
  (see Figure \ref{fig:m1p0_profs} Time 2, left plot). As $\Tc$ continues to decrease, the temperature profile inverts and the center is no longer the hottest region in the star. Eventually, $\Tc < \Tx$ and dark matter begins moving energy back towards the center ($\epsx(r=0) > 0$, see Time 3 in Figure \ref{fig:m1p0_profs} \todo{possibly pick this time a little later, after xheat at center >0}). This increases both the central temperature and burning rate until $\Tc > \Tx$ and the cycle starts over. The effects propagate out to the surface where
  $L$, $\Teff$, and $\Rstar$ all oscillate in response to changes in the core. Similar oscillatory behavior was noted in \citet{Iocco2012} \qus{seems like I need to say more here, but I'm not sure what. The Iocco paper (arXiv:1201.5387v1) states the following: "In spite of our efforts, we have not been able to fully understand the reason of the oscillations seen in Figure 3, whether they are a physical effect arising from the “bouncing” of the central temperature on the WIMP temperature floor, or a numerical artifact. We have however checked that the existence of oscillations does not affect our results nor does change our conclusions. Beside the theoretical con- sistency of the intepretation presented in the paper, one can be convinced of the actual physical consistency of our results with observations based on several numerical experiments we have performed: ..."}.

  The process repeats, with increasing amplitude, until the central temperature falls low enough that the core becomes significantly degenerate. Once electron degeneracy pressure can support the core, the temperature decouples from the equation of state and the star quickly settles into a more stable configuration with a significant temperature inversion (see Time 4 in Figures \ref{fig:m1p0_profs} and \ref{fig:m1p0_kipp}).

  Overall, the ADM captured by these stars causes burning in a larger volume and at a higher rate than in standard models (see Figure \ref{fig:m1p0_kipp}). The stars then burn through the central hydrogen supply faster and leave the MS earlier: by 2.5 Gyr, solar mass stars with $\gbpow{6}$ have already left the MS and are climbing the red-giant branch. Note that the 2.5 Gyr, $\gbpow{6}$ isochrone is most similar the 10 Gyr, NoDM isochrone (Figure \ref{fig:isos}).


  Notes:
  hydrogen depletes in shell first.
  % \tjr{I'm not totally sure this is the CAUSE of the oscillations, but I've checked a lot of other things and ruled them out. Lower cboost models sometimes transport more energy than produced by burning, but never more than a few erg/g/s. c5 and c6 models transport a minimum of ~5 erg/g/s more than is generated from burning and usually much more than that (a few hundred erg/g/s during oscillations, up to 1000 erg/g/s by the time it goes degenerate).}

  % \tjr{burning changes before temperature which I don't really understand. See figure \ref{fig:osc_scaled_values} for more info. Order of changes seems to be: extra heat -> density -> burning -> temperature. Tried to check how mesa calculates the burning rate to find dependencies other than temperature, but the calculation seems pretty scattered and complicated and I couldn't find anything useful.}

  Possible work for future: 1. vary DM mass and cross section (given observational constraints) to see how quick the effects vanish. 2. model iso curves and quantify how much older they look.


  %--------------------


  %--------------------
  % % m1p0 c0 kipp
  % \begin{figure*}
  % \centering
  %   \includegraphics[width=\textwidth]{plots/m1p0c0_kipp.png}
  %   \caption{\label{fig:m1p0c0_kipp}Star mass 1.0Msun, c0, Main sequence burning. Note that the burning (shaded in red, left axis) doesn't extend as far as c6 model.}
  % \end{figure*}
  %--------------------

  %  \tjr{why are they destabilized? centerRho is higher, centerT is lower, Tmax-Tc is lower than lower cboosts (except extreme oscillations), oscillations start before Tc-Tx<0. what makes it go from transporting energy away from center to transporting to center? temp profile flattens (inverts? but cb<5 also invert so why are they still stable? they transport more energy than epsnuc. yes, oscillations turn on more slowly in c5, stable at $\epsilon_{\rm{x}}(r=0) \approx \epsilon_{\rm{nuc}}(r=0)$ for beginning of MS)}

  % The PP chain is less efficient? NO! PP, CNO have same net energy per pp fusion.

  % Nx and nxcenter are higher here than high mass stars, but wimps transport less heat here. why? Tx-Tc is much lower. but they still have a larger effect than in high mass stars because there is less energy flux to carry.

  % --------
  % 1.0 pgstar notes:
  % what i think is happening:
  % temp is lower so burning rate is much lower.
  % wimps take kinetic energy from protons. transport all of the energy generated from burning plus additional kinetic energy. burning rate goes down. temperature goes down.
  % temp and burning rate are still high (higher than c0?) in a shell where wimps have been depositing the energy. temp inverts and wimps start moving energy back to core (largest deltaT just before this happens?). proton kinetic energy goes up (so they are more likely to fuse when colliding) and burning rate goes up (largest deltaT here? no.).
  % 2 energy sources increase so temp goes up. temp no longer sufficiently inverted, wimps start moving energy away from core and process starts over.

  % stabalizes with large T inversion (deltaT=7.13-7.19 =4e-2) when degeneracy becomes important.

  % at center unless otherwise specified:
  % General oscillation pattern:
  % xheat negative and decreasing
  % burning, temp, and R decreasing, deltaT O(5e-4)
  % xheat peaks low, stays close for awhile

  % at 58 secs:
  % xheat, burning, temp all decreasing
  % TMR increasing, R decreasing

  % xheat peaks low at -80 (deltaT=7.0660-7.0656= 4e-4) then goes up
  % burning peaks low at 11 (TMR peaks high at same time or slightly before) then goes up
  % T peaks low at 7.0567 (deltaT=7.0573-7.0567 =6e-4) then goes up

  % xheat increases past 0 (deltaT=7.0582-7.0576= 6e-4)
  % luminosity goes negative at 0.08Msun
  % deltaT increasing

  % xheat peaks at 130 (deltaT=7.0711-7.0695= 16e-4) then oscillates up and down a bit
  % after last xheat peak, burning peaks at 20 then goes down
  % deltaT decreasing

  % xheat decreases past 0 (deltaT=7.0777-7.0771 =6e-4)
  % T peaks at 7.0772 (deltaT=7.0777-7.0772 =5e-4) then goes down
  % luminosity to zero (non-negative)

  % xheat peaks low at -100 (deltaT=7.0725-7.0722 =3e-4) then oscillates up and down a bit
  % after last xheat peak, burning peaks low at 11 then goes up
  % T peaks low at 7.0553 (deltaT=7.0558-7.0553 =5e-4)

  % xheat increases past 0 (deltaT=7.0569-7.0562 =7e-4)
  % luminosity goes negative at 0.08Msun

  % earlier:
  % xheat, burning, temp all decreasing
  % xheat peaks low at -40 then goes up
  % burning peaks low at 12 then goes up
  % T peaks low at 7.0592 then goes up

  % burning peaks at 14.5
  % xheat crosses 0, decreasing
  % T peaks high at 7.0664
  % burning decreasing
  % temp decreasing

  % temp is inverted
  % burning increasing
  % xheat positive
  % core radius is shrinking
  % R increasing

  % center temp increases, still inverted
  % xheat positive and decreasing
  % burning peaks
  % xheat goes negative
  % burning decreases

  % 1.0 pgstar notes end
  % --------


  % Stars leave the main sequence at a lower luminosity, which continues through the sub-giant branch (since He core mass is lower?), and the convective hook disappears. Caused by: lower gradT => lower central burning (CNO temp dependence) => core convection turns off => transition to H shell burning is more gradual.


\section{Conclusions}
\label{sec:conclusions}

  \arz{I think that most of your conclusions section could be (1) a summary of our results and (2) speculations on how to turn this into a constraint in the future.}

\section*{Acknowledgements}



%%%%%%%%%%%%%%%%%%%%%%%%%%%%%%%%%%%%%%%%%%%%%%%%%%







%%%%%%%%%%%%%%%%%%%% REFERENCES %%%%%%%%%%%%%%%%%%

% The best way to enter references is to use BibTeX:

\bibliographystyle{mnras}
\bibliography{references}

%%%%%%%%%%%%%%%%%%%%%%%%%%%%%%%%%%%%%%%%%%%%%%%%%%

%%%%%%%%%%%%%%%%% APPENDICES %%%%%%%%%%%%%%%%%%%%%

% \appendix

% \section{section}
% \label{sec:sec}



%%%%%%%%%%%%%%%%%%%%%%%%%%%%%%%%%%%%%%%%%%%%%%%%%%



% Don't change these lines
\bsp	% typesetting comment
\label{lastpage}
\end{document}

% End of mnras_template.tex
